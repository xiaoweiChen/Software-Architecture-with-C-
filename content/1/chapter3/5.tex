
完成了前面描述的步骤之后,就可以将收集到的需求放在一个文档中进行细化。文档采用什么形式以及何管理它并不重要,重要的是要有一份文档,将所有相关方放在同一页上,说明产品需要什么,以及每个需求将带来什么价值。

需求是由所有相关方产生和消费的,他们中的大部分人需要阅读文档。这意味着架构师应该编写这份文档,以便能为具有各种技术技能的人带来价值,从客户、销售人员和市场人员,到设计师和项目经理,再到软件架构师、开发人员和测试人员。

有时可以准备两个版本的文档,一个是为最接近项目业务方面的人,另一个是为开发团队,更技术性的版本。通常,只要编写一份能看懂的文档就足够了,其中的章节(有时是单个段落)或整章都会有更多技术细节。

现在来看看哪些部分需要进入需求文档。

\subsubsubsection{3.5.1\hspace{0.2cm}记录背景}

需求文档应该作为进入项目的入口点之一,应该概述产品的目的,谁将使用它,以及如何使用。在设计和开发之前,产品团队成员应该阅读文档,以清楚地知道他们要做些什么。

背景部分应该提供系统的概述——为什么要构建,试图实现什么业务目标,以及将交付什么关键功能。 

可以描述一些典型的用户角色,例如首席技术官John,或者司机Ann,使读者可以把系统的用户当作真实的人来思考,并理解对他们的期望。

在了解背景一节中描述的所有内容也应该有总结的部分,有时甚至要在文档中设立单独的部分。背景和范围部分应该提供大多数非项目相关方所需的所有信息,这些信息应该简洁和精确。

对于任何想要研究并决定的开放性问题也是如此。对于你所做的每一个决定,最好注意以下几点:

\begin{itemize}
\item 
这个决定本身是什么

\item 
谁来做,何时做

\item 
这样做的理由是
\end{itemize}

现在已经了解了如何记录项目的背景,继续了解如何正确地描述范围。

\subsubsubsection{3.5.2\hspace{0.2cm}记录范围}

这个部分应该定义什么在项目范围内,什么是在项目范围外。应该提供一个基本准则,说明为什么要以特定的方式定义范围,特别是在编写一些不符合要求的内容时。

本节还应该介绍高级功能和非功能性需求,但详细信息应在本文档的后续部分中介绍。如果熟悉敏捷实践,请在此描述一些传奇和大用户的故事。

如果架构师或相关方对范围有假设,应该在这里提到这些假设。如果由于问题或风险,范围可能会发生变化,那么也应该写一些相关的文字,以及必须做出的权衡。

\subsubsubsection{3.5.3\hspace{0.2cm}记录功能性需求}

每个需求都应该是精确和可测试的。考虑这个例子:“系统将为司机提供一个排名系统。”如何创建针对它的测试?最好是为排名系统创建一个章节,并在那里精确的定义其需求。

考虑另一个例子:如果有一位免费司机离乘客很近,应该告知他们即将到来的乘车请求。如果有多个合适的司机怎么办?我们能描述的最近距离是多少?

此需求既不精确,又缺乏业务逻辑。我们只能希望没有可用司机的情况是另一个需求。

2009年,Rolls Royce开发了\textbf{简单需求方法语法(EARS)}来帮助解决这个问题。在EARS中,有五种基本类型的需求,它们以不同的方式编写并服务于不同的目的,可以将它们组合起来创建更复杂的需求。这些基本的问题如下:

\begin{itemize}
\item 
\textbf{无处不在型需求}: "\textit{\$SYSTEM}应是\textit{\$REQUIREMENT}," 例如,应用将使用C++开发。

\item 
\textbf{事件驱动型需求}: "当\textit{\$TRIGGER} 是\textit{\$OPTIONAL$\backslash$\_PRECONDITION}时,\textit{\$SYSTEM}
应是\textit{\$REQUIREMENT}," 例如:当订单到达时,网关将生成一个NewOrderEvent。

\item 
\textbf{意外行为型需求}: "若是\textit{\$CONDITION},则\textit{\$SYSTEM}应是\textit{\$REQUIREMENT}," 例如:如果处理请求的时间超过1秒,该工具将显示一个进度条。

\item
\textbf{状态驱动型需求}: "当为\textit{\$STATE}时,\textit{\$SYSTEM}应是\textit{\$REQUIREMENT}," 例如:在开车时,应用会显示一张地图,帮助司机导航到目的地。

\item
\textbf{自选特性}: "在具有\textit{\$FEATURE}时,\textit{\$SYSTEM}应是\textit{\$REQUIREMENT}," 例如:在有空调的地方,App会让用户通过手机设置温度。
\end{itemize}

更复杂的需求示例是:在使用双服务器设置时,如果备份服务器在5秒内没有收到主服务器的消息,应该尝试将自己注册为一个新的主服务器。

也可以不使用EARS,但是如果与不明确的、模糊的、过于复杂的、不可测试的、省略的或措词糟糕的需求作过斗争,EARS可以提供帮助。无论选择何种方式或措辞,都要确保使用简洁的模型,该模型基于通用语法并使用预定义关键字。为列出的每一个需求分配一个标识符也是一个很好的实践方式,这样就可以使用一种简单的方法来引用它们。

当涉及到更详细的需求格式时,应该有以下字段:

\begin{itemize}
\item 
\textbf{ID或索引}: 便于识别特定需求。

\item 
\textbf{标题}: 可以在这里使用EARS模板。

\item 
\textbf{详细描述}: 可以将相关的信息放在这里,例如:用户故事。

\item
\textbf{拥有者}: 这个要求服务于谁,可能是产品所有者、销售团队、法律、IT等。

\item
\textbf{优先级}: 不言自明。

\item
\textbf{交付方式}: 如果关键日期都有要求,可以在这里注明。
\end{itemize}

既然已经了解了如何记录功能性需求,继续看一下应该如何记录非功能性需求。

\subsubsubsection{3.5.4\hspace{0.2cm}记录非功能性需求}

每个质量属性,例如性能或可扩展性,都应该在文档中有记录,并列出特定的、可测试的需求。大多数都是QA可测试的,所以拥有特定的测试标准可以很好地解决问题。当然,还可以用单独的部分来说明项目的约束条件。

关于措辞,可以使用EARS模板来记录您的NFR(\textbf{N}on\textbf{F}unctional \textbf{R}equirements)。或者,使用在背景信息中定义的角色,将其指定为用户故事。

\subsubsubsection{3.5.5\hspace{0.2cm}管理文档版本的记录}

可以采用以下两种方法的一种:在文档内创建版本日志或使用外部版本控制工具。这两种方法各有利弊,但建议采用后一种方法。就像对代码使用版本控制系统一样,也可以将其用于文档。并不是说必须使用存储在Git库中的Markdown,只要生成一个\textbf{适用于业务人群}的文档,无论是网页还是PDF文件,这都是很好的方法。或者,也可以使用在线工具,比如RedmineWikis或Confluence,这些页面允许在每次发布的编辑中添加有意义的评论,描述修改的内容,并查看版本之间的差异。

如果决定用修订日志的方法,表中通常是包含以下字段:

\begin{itemize}
\item 
\textbf{修改}: 一个数字,标识变更是在文档的哪个迭代中引入的。如果愿意,还可以为特殊修订添加标签,例如\textit{初稿}。

\item 
\textbf{更新}: 谁做的改变。

\item 
\textbf{审核}: 谁审阅的修改。

\item
\textbf{修改描述}: 修改的\textit{提交信息}。用来说明了发生了什么变化。
\end{itemize}

\subsubsubsection{3.5.6\hspace{0.2cm}记录敏捷项目中的需求}

许多敏捷的支持者声称,记录所有的需求纯粹是浪费时间,因为它们可能都会发生变化。然而,一个好的方法是将它们与待办事项列表中的项目进行类似的处理:在即将到来的sprint中开发的项目,应该比稍后实现的项目定义得更详细。如果不愿意在必要的时候将史诗巨作分解成故事和任务,那么可以在确定需要实现任务之前,对粗粒度需求只进行概要性的描述。

\begin{tcolorbox}[colback=webgreen!5!white,colframe=webgreen!75!black, title=TIP]
\hspace*{0.7cm}确定需求的来源,这样就可以知道哪里会提供输入,以便在将来对其进行细化。
\end{tcolorbox}

以多米尼加博览会为例。在下一个sprint中,将构建供访问者查看的商店页面,再下一个sprint中,将添加订阅机制。需求如下所示:

\begin{table}[H]
	\begin{tabular}{|l|l|l|l|}
		\hline
		\textbf{ID} & \textbf{优先级} & \textbf{描述}                                                                                                            & \textbf{负责人}                                   \\ \hline
		DF-42       & P1                & \begin{tabular}[c]{@{}l@{}}商店页面必须可以显示商店的库存,并附带有\\照片和每个商品的价格。\end{tabular} & Josh, Rick                                              \\ \hline
		DF-43       & P2                & \begin{tabular}[c]{@{}l@{}}商店页面必须在地图上标明有线下店铺的位置。\end{tabular}                          & \begin{tabular}[c]{@{}l@{}}Josh,\\ Candice\end{tabular} \\ \hline
		DF-44       & P2                & 客户可以通过商店页面订阅商店。                                                                        & Steven                                                  \\ \hline
	\end{tabular}
\end{table}

前两项与接下来要做的特性有关,所以它们描述得更详细。不过,谁知道呢,也许在下一个sprint之前,关于订阅的要求会取消,所以考虑其细节信息也没有意义。

在有些情况下,仍然需要拥有一个完整的需求列表。如果需要与外部监管机构或内部团队(如审计、法律或合规)打交道,可能需要提供编写良好的文档。有时候,只给他们一个包含从backlog中提取的工作项文档就可以了。最好与此类负责人沟通清楚:收集其期望,以了解满足需求,以便最小化文档。

编写需求文档的重要之处在于,和提出特定需求的各方之间要互相理解。如何实现这一点?当准备好了草稿,就应该向他们展示文档并收集反馈。这样,就会知道什么是模糊的,不清楚的,或缺失的。即使需要一些迭代,也将帮助与众负责人建立一个种共识,从而获得更多的信心,并相信正在构建正确的东西。

\subsubsubsection{3.5.7\hspace{0.2cm}其他部分}

有一个链接和资源部分是一个好主意,在其中你可以指向诸如问题跟踪板、工件、CI、代码库,以及任何可以陈现的东西。架构、营销和其他类型的文档也可以在这里列出。

如果需要,还可以包含词汇表。

现在知道了如何记录需求和相关信息。接下来,就简单介绍一下如何为设计的系统编写文档。


