
创建软件系统时,应该不断地问自己,所做的是否是客户所需要的。很多时候,客户不知道如何才能能满足自己的需求。架构师的角色是发现产品的需求,并确保需求得到满足。这里需要考虑三种不同类型的需求:功能需求、质量属性和约束条件。

\subsubsubsection{3.2.1\hspace{0.2cm}功能需求}

第一组是功能需求。这定义了系统应该做什么,应该提供什么功能。

\begin{tcolorbox}[colback=webgreen!5!white,colframe=webgreen!75!black, title=TIP]
\hspace*{0.8cm}功能并不总是影响体系结构,因此必须注意哪些需求将影响解决方案。
\end{tcolorbox}

通常,功能需求具有某些必须满足的特性,那么在架构上就很重要。考虑为参加多米尼加博览会(Dominican Fair)的商人和游客开发一款应用程序,该博览会是在Gdańsk举办的年度活动,内容包括音乐、各种艺术和商店。下面是一些功能需求的例子:

\begin{itemize}
\item 
\textit{作为一个店主,想过滤包含特定产品的订单。}

\item 
\textit{单击“订阅”按钮,将客户添加到选定商家的通知列表中。}
\end{itemize}

第一个要求,必须有一个具有搜索功能的跟踪订单和产品的组件。根据UI的显示方式和应用的规模,可以只在应用中添加一个简单的页面,或者可能需要Lucene(一个全文搜索引擎)或Elasticsearch(搜索引擎解决方案)等特性。这意味着\textbf{体系结构重要需求(ASR)},会影响架构。

第二个例子更直接,我们知道需要订阅和发送通知的服务。这无疑是架构上的重要功能需求。现在来看一些可以是ASR的\textbf{非功能需求(NFR)}。

第一个需求实际上是以用户故事的形式给出的。用户故事是以以下格式给出的需求:“\textit{作为一个<角色>,我可以/想要<能力>,从而可以<受益> 。}这是一种表达需求的常用方法,可以帮助相关负责人和开发人员找到共同点,从而更好地进行沟通。

\subsubsubsection{3.2.2\hspace{0.2cm}非功能需求}

非功能需求关注的不是系统应有什么功能,而是系统应该如何,以及在何种条件下执行所述功能。这是由两个主要的需求组成:\textbf{质量属性(QA)}和\textbf{约束条件}。

\hspace*{\fill} \\ %插入空行
\noindent
\textbf{质量属性}

\textbf{质量属性(QA)}是解决方案的特征,例如:性能、可维护性和用户友好性。软件可以具有几十种(如果不是数百种的话)不同的质量。在选择软件包含的功能时,试着只关注那些重要的功能,而不是所有出现在脑中的功能。质量属性需求的例子如下所示:

\begin{itemize}
\item 
正常负载下,系统将在500ms以内对99.9\%的请求作出响应(不要忘记定义“正常负载”)。

\item 
网站不会存储在支付过程中泄漏的客户信用卡数据(保密)。

\item
更新系统时,如果更新任何组件失败,系统将回滚到更新之前的状态(生存性)。

\item 
作为Windows、macOS和Android的用户,希望能够同时使用这些系统(可移植性。理解需求是否需要支持桌面、移动和/或Web等平台)。
\end{itemize}

虽然在列表中捕捉功能需求非常简单,但不能对质量属性需求进行同样的操作。幸运的是,有其他方法可以做到这一点:

\begin{itemize}
\item 
有一些可以用\textbf{完成的定义}或\textbf{验收标准}来表示任务、故事和发布。

\item 
其他的可以直接表示为用户故事(如前面的最后一个示例所示)。

\item 
还可以将它们作为设计和代码审查的一部分进行检查,并为其中一些创建自动化测试。
\end{itemize}

\hspace*{\fill} \\ %插入空行
\noindent
\textbf{约束条件}

约束条件是在交付项目时必须遵循的决策。这些决策可以是设计决策、技术决策,甚至是政治决策(关于人员或组织事务)。另外两个常见的约束是\textbf{时间}和\textbf{预算}。约束条件的例子有:

\begin{itemize}
\item 
\textit{团队的规模永远不会超过四名开发人员、一名QA工程师和一名系统管理员。}

\item 
\textit{由于我们公司在其所有现有产品中都使用了Oracle DB,所以新产品也必须使用,这样才能充分利用我们的专业知识。}
\end{itemize}

非功能性需求总是会影响架构。重要的是不要过度指定,因为在产品开发过程中,假阳性将是一种持续性负担。同样重要的是,不要对它们要求不足,因为这可能会导致错过销售时机或未能遵守监管机构的要求。

下一节中,将了解如何在这两个极端之间取得平衡,并只关注在特定情况下那些真正重要的需求。





