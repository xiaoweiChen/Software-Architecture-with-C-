
架构是一个过于复杂的主题,无法用一个大图表来描述。假设一幢建筑的建筑师,为了设计整个建筑,需要不同视角的独立图表:一个用于设计管道,另一个用于部署电力和其他电缆等。每个图都显示了项目的不同方面。软件架构也是如此:需要从不同的角度,针对不同的受众来展示软件。

此外,如果正在建造一座智能楼宇,很有可能需要规划出想要放置的设备的计划。尽管不是所有的项目都需要,但在当前的项目中需要,所以需要添加。同样的方法也适用于架构:如果发现一个不同的视图对文档有价值,那么就应该这样做。那么,如何知道哪些视图可能是有价值的呢?您可以尝试以下步骤:

\begin{enumerate}
\item
从4+1模型或C4模型的视图开始。

\item 
询问您的相关方,对他们文档化来说什么是必要的,并考虑修改视图集。

\item
选择可以评估体系结构是否满足其目标,以及是否满足所有ASR的视图。阅读下一节中每个视图的第一段,检查它们是否适配需要。
\end{enumerate}

If you're still not sure which views to document, here's a set of tips:

\begin{tcolorbox}[colback=webgreen!5!white,colframe=webgreen!75!black, title=TIP]
\hspace*{0.7cm}尝试只选择最重要的视图,因为当它们太多时,体系结构将变得难以理解。好的视图不仅应该展示架构,还应该向项目暴露技术风险。
\end{tcolorbox}

选择应该在文档中描述哪些视图时,需要进行考虑。在这里简要地描述它们,但如果感兴趣,可以在\textit{扩展阅读}部分中找到\textit{Rozanski和Woods}的书,并进行阅读。

\subsubsubsection{3.7.1\hspace{0.2cm}功能视图}

若软件是作为更大的系统的一部分开发,特别是与不进行日常交流的团队一起开发,那么就应该包括一个功能视图(就像在4+1模型中那样)。

在为架构编写文档时,接口的定义是最需要描述的事情,也是经常忽视的。无论是两个组件之间的接口,还是外部世界的入口点,都应该花时间进行完备地记录,描述对象和调用的语义,以及使用示例(有时可以作为测试重用)。

文档中包含功能视图的另一个好处是,阐明了系统组件之间的职责。每个开发系统的团队都应该了解边界在哪里,谁负责开发哪些功能。所有需求都应该显式地映射到组件,以消除差距和重复的工作。

\begin{tcolorbox}[colback=webgreen!5!white,colframe=webgreen!75!black, title=TIP]
\hspace*{0.7cm}这里需要注意的事情是避免重载函数视图。如果弄得一团糟,就没人想看了。如果开始仅在视图上描述基础设施,那么可以考虑添加一个部署视图。如果在模型中有一个上帝对象,试着重新思考设计,把它分成更小的,更内聚的组件。
\end{tcolorbox}

关于功能视图的最后一个注意事项:尽量将包含的每个图维持在同一个抽象级别上。另一方面,不要选择过于抽象的层次而使它过于模糊,从而确保相关方对每个元素都能正确的定义和理解。

\subsubsubsection{3.7.2\hspace{0.2cm}信息视图}

如果系统对信息、处理流、管理过程或存储有非直接的需求,那么需要包含这种视图。

选取最重要的、数据丰富的实体,展示如何在系统中流动,谁拥有它们,以及谁是生产者和消费者。标记某些数据的保质期,以及何时可以安全地丢弃,到达系统的特定点的预期延迟是什么,或者如果系统工作在分布式环境中,如何处理标识符,这些可能是有用的。如果系统管理事务,则相关方也应该清楚这个过程以及任何回滚。转换、发送和持久化数据的技术也很重要。如果从事金融领域的业务或必须处理个人数据,很可能需要遵守一些规定,因此请描述系统计划如何解决这一问题。

数据的结构可以用UML类模型来表示。要清楚数据的格式,特别是在两个不同的系统之间交互时。这是它与洛克希德·马丁公司共同开发的,因为他们在不知情的情况下使用了不同的单位,从而导致NASA失去了价值1.25亿美元的火星气候轨道器,所以要注意系统之间的数据不一致。

数据的处理流程可以使用UML的活动模型,并且可以使用状态图来显示信息的生命周期。

\subsubsubsection{3.7.3\hspace{0.2cm}并发视图}

如果运行许多并发执行单元对产品很重要,请考虑添加并发视图。它可以显示可能遇到的问题和瓶颈(除非听起来太详细)。使用它的其他原因是依赖于进程间通信、具有非直接的任务结构、并发状态管理、同步或任务失败处理逻辑。

可以为这个视图使用想要的表示法,只要它可以捕获执行单元及其通信。如果需要,为进程和线程分配优先级,然后分析任何潜在问题,例如死锁或争用。可以使用状态图来显示可能的状态以及重要执行单元(等待查询、执行查询、分发结果等)的转换。

如果不确定是否需要将并发性引入系统,好的经验法则是\textit{不要做}。如果必须的话,尽量设计得简单一些。调试并发性问题从来都不是一件容易的事情,而且时间花费总是很长。因此,可能的话,首先尝试优化现有的问题,而不是抛出更多的线程来处理手头的问题。

查阅图后,若担心资源争用,可以尝试用更多的锁替换大对象上的锁,这样的粒度会更细,使用轻量级同步(有时原子就足够了),引入乐观锁,或者减少共享的内容(在线程中创建一些数据的外副本,并对其进行处理可能比共享访问唯一副本更快)。

\subsubsubsection{3.7.4\hspace{0.2cm}发展视图}

如果正在构建一个包含许多模块的大型系统,并且需要结构化代码,拥有系统范围的设计约束,或者如果想在系统的各个部分之间共享,那么从开发的角度展示解决方案应该会使架构师,软件开发人员和测试人员更了解这个系统。

开发视图的包图可以明确地展示系统中不同的模块位于何处,依赖关系是什么,以及其他相关的模块(例如,驻留在同一个软件层)。不需要是UML图—即使是框和行也可以。如果计划一个模块是可替换的,这种图表可以显示出有哪些软件包可能会受到影响。

增加系统重用的策略,例如:为组件创建运行时框架,或者增加系统一致性策略。认证、日志、国际化或其他类型的处理的通用方法,都是开发视图的一部分。可以将系统可见的公共部分记录下来,以确保所有开发人员也能看到。

通用的代码组织、构建和配置管理方法,也应该出现在文档中。如果所有这些听起来都需要记录,那么就专注于最重要的部分,并简要介绍其余部分。

\subsubsubsection{3.7.5\hspace{0.2cm}部署和操作视图}

如果有一个非标准的或复杂的部署环境,例如关于硬件、第三方软件或网络需求的特定需求,针对系统管理员、开发人员和测试人员,可以考虑将其记录在一个单独的部署部分中。

如有必要,应包括下列内容:

\begin{itemize}
\item
所需内存

\item 
CPU线程数(有或没有超线程)

\item 
关于NUMA节点的亲和性

\item 
专业网络设备的要求,如交换机标记

\item 
以黑盒方式度量延迟和吞吐量的包

\item 
网络拓扑结构

\item 
估计所需带宽

\item 
应用程序的存储要求

\item 
计划使用到的第三方软件
\end{itemize}

有了需求,就可以将它们映射到特定的硬件,并将它们放入运行时平台模型中。如果想要正式建模,可以使用带有构造型的UML部署图。这应该显示处理节点和客户机节点、在线和离线存储、网络链接、专门的硬件(如防火墙、FPGA或ASIC设备),以及功能元素和运行节点之间的映射。

如果有非直接的网络需求,可以添加另一个图来显示网络节点和它们之间的连接。

如果依赖于特定的技术(包括软件的特定版本),最好将它们列出来,以查看所使用的软件是否存在兼容性问题。有时,两个第三方组件需要相同的依赖关系,但版本不同。

如果头脑中有一个特定的安装和升级计划,写几句话来说明可能是个好主意。解决方案所依赖的A/B测试、蓝绿部署或特定容器的使用技巧,相关人员都很应该很清楚。如果需要,还应该包括数据迁移计划,包括迁移需要多长时间,以及何时可以安排迁移。

配置管理、性能监视、操作监视和控制,以及备份策略的计划都值得描述。可能需要创建几个组,确定每个组的依赖关系,并为每个组定义方法。如果能够考虑到可能发生的错误,就制定一个计划来检测,并尝试从错误中恢复。

对支持团队的注意事项也可以进入这个部分:哪个相关方需要什么支持,计划有什么类别的事件,如何升级,以及每个级别的支持将负责什么。

最好尽早与一线人员接触,并为他们专门创建图表,以保持他们的参与度。

已经讨论了如何手动创建关于系统及其需求的文档,接下来切换到以自动化的方式为API编写文档。









