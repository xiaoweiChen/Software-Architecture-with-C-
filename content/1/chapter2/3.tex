用于开发的最简单架构风格是整体架构,这就是为什么许多项目开始使用这种风格的原因。相应的应用程序是一个大块,这意味着应用程序中可区分的功能(如处理I/O、数据处理和用户界面)是相互交错的,而不是位于架构组件中。这种架构风格的例子是Linux内核,内核是整体的并不妨碍它进行模块化。

与部署多组件相比,因为只需要部署一个组件,所以部署单组件可能更容易。并且单个组件更容易测试,因为端到端测试只需要启动单个组件。集成也更容易,在扩展解决方案的同时,可以在负载均衡器后面添加更多的实例。有这么多优点,那为什么还会有人害怕这种架构风格呢?事实证明,尽管优点不少,但缺点也很多。

所提供的扩展性从理论上看起来不错,但如果模块有不同的资源需求该怎么办?如果只需要扩展应用程序中的一个模块呢?缺乏模块化(整体系统的固有属性)是此架构许多缺陷的根源。

而且,开发整体应用程序的时间越长,维护时遇到的问题就越多。因为在模块之间添加另一个依赖关系非常容易,所以维持这种内部松散的耦合着实是一个挑战。随着应用程序的成长,会变得越来越难以理解。由于增加了复杂性,开发过程很可能会随着时间的推移而变慢。在进行整体开发时,维护\textbf{设计驱动的开发(DDD, Design-Driven Development)}的上下文边界也很难定界定。

大应用在部署和执行方面也有缺点,启动应用所需的时间要比启动更多、更小的服务的时间长得多。无论在应用中做了怎样的更改,可能都不想因其立即重新部署整个应用。现在,假设一个开发人员在应用中引入了一个资源泄漏。如果漏洞代码反复执行,不仅会影响应用程序的功能,还会影响应用的其他部分。

如果喜欢在项目中使用前沿技术,那么整体化的风格就不那么搭了。由于现在需要一次性迁移整个应用,因此更新库或框架都十分困难。

整体架构只适合于简单和小型的应用程序。然而,在另外一种情况下,使用它可能非常合适。如果关心性能,那么与微服务相比,大块的应用有时可以获得更多的延迟或吞吐量。进程间通信总是会产生一些开销,而整体应用不需要支付这些开销。如果对测量结果感兴趣,请参阅本章\textit{扩展阅读}部分中的文章。






















