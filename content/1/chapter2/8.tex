本章中,讨论了各种架构风格,都可以应用到软件中。我们已经讨论了整体架构,讨论了面向服务的架构。然后,转向了微服务,并讨论了提供的外部接口和交互的各种方法。也了解了如何编写REST式服务,以及如何创建一个有自愈性,且易于维护的微服务体系结构。

还展示了如何创建简单的客户机来使用同样简单的服务。之后,讨论了构建架构的各种其他方法:事件驱动的方法、基于运行时模块的方法,并展示了分层的位置和原因。现在,我们已经了解了如何实现事件源,以及识别何时使用BFF。现在明白了架构风格是如何实现一些质量属性的,以及这会带来怎么样的挑战。

下一章中,将了解如何知道在给定的系统中,哪些属性更重要。


