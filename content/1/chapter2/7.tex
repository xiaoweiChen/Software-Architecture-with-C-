\begin{tcolorbox}[colback=blue!5!white,colframe=blue!75!black, title=Note]
\hspace*{0.7cm}In this section, by modules, we mean software components that can be loaded and unloaded in runtime. For C++20's modules, refer to Chapter 5, Leveraging C++ Language Features.
\end{tcolorbox}

If you've ever needed to run a component with as little downtime as possible, but for any reason couldn't apply the usual fault-tolerance patterns, such as redundant copies of your service, making this component module-based can come to save your day. Or you may just be attracted by a vision of a modular system with versioning of all the modules, with an easy lookup of all the available services, along with the decoupling, testability, and enhancing teamwork that module-based systems can cause. All of this is why \textbf{Open Service Gateway Initiative (OSGi)} modules were created for Java and got ported to C++ in more than a few frameworks. Examples of architectures using modules include IDEs such as Eclipse, \textbf{Software Defined Networking (SDN)} projects such as OpenDaylight, or home automation software such as OpenHAB.

OSGi also allows for automatic dependency management between modules, controlling their initialization and unloading, as well as controlling their discovery. Since it's  serviceoriented, you can think of using OSGi services as something akin to having tiny (micro?) services in one "container". This is why one of the C++ implementations is named C++ Micro Services. To see them in action, refer to their Getting Started guide from the Further reading section.

One interesting concept adopted by the C++ Micro Services framework is a new way to deal with singletons. The \textit{GetInstance()} static function will, instead of just passing a static instance object, return a service reference obtained from the bundled context. So effectively, singleton objects will get replaced by services that you can configure. It can also save you from the static deinitialization fiasco, where multiple singletons that depend on each other have to unload in a specific order.





