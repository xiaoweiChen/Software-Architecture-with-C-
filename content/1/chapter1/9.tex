本章中,讨论了什么是软件架构,以及为什么值得关注。展示了当架构没有随着需求和实现的变化而更新时会发生什么,以及如何在敏捷环境中看待架构。然后,将注意力转向C++语言的一些核心原则。

因为C++不仅编写面向对象的代码,还可以编写非面向对象代码,所以许多来自软件开发的术语在C++中可以有不同的理解。最后,讨论了耦合和内聚等术语。

现在,应该能够在代码审查中指出许多设计缺陷,并重构的解决方案,以获得更高的可维护性,同时作为开发人员,更不容易出现Bug。并且,可以设计健壮、自解释和完整的类接口。

下一章中,将了解不同的架构方法或风格,以及何时使用它们来获得更好的结果。