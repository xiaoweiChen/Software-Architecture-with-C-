
It's important to know how to recognize a good architecture from a bad one, but it's not an easy task. Recognizing anti-patterns is an important aspect of it, but for an architecture to be good, primarily it has to support delivering what's expected from the software, whether it's about functional requirements, attributes of the solution, or dealing with the constraints coming from various places. Many of those can be easily derived from the architecture context.

\subsubsubsection{1.4.1\hspace{0.2cm}Architecture context}

The context is what an architect takes into account when designing a solid solution. It comprises requirements, assumptions, and constraints, which can come from the stakeholders, as well as the business and technical environments. It also influences the stakeholders and the environments, for example, by allowing the company to enter a new market segment.

\subsubsubsection{1.4.2\hspace{0.2cm}Stakeholders}

Stakeholders are all the people that are somehow involved with the product. Those can be your customers, the users of your system, or the management. Communication is a key skill for every architect and properly managing your stakeholder's needs is key to delivering what they expected and in a way they wanted.

Different things are important to different groups of stakeholders, so try to gather input from all those groups.

Your customers will probably care about the cost of writing and running the software, the functionality it delivers, its lifetime, time to market, and the quality of your solution.

The users of your system can be divided into two groups: end users and administrators.The first ones usually care about things such as the usability, user experience, and performance of the software. For the latter, more important aspects are user management, system configuration, security, backups, and recovery.

Finally, things that could matter for stakeholders working in management are keeping the development costs low, achieving business goals, being on track with the development schedule, and maintaining product quality.


\subsubsubsection{1.4.3\hspace{0.2cm}Business and technical environments}

Architecture can be influenced by the business side of the company. Important related aspects are the time to market, the rollout schedule, the organizational structure, utilization of the workforce, and investment in existing assets.

By technical environment, we mean the technologies already used in a company or those that are for any reason required to be part of the solution. Other systems that we need to integrate with are also a vital part of the technical environment. The technical expertise of the available software engineers is of importance here, too: the technological decisions an architect makes can impact staffing the project, and the ratio of junior to senior developers can influence how a project should be governed. Good architecture should take all of that into account.

Equipped with all this knowledge, let's now discuss a somewhat controversial topic that you'll most probably encounter as an architect in your daily work.















