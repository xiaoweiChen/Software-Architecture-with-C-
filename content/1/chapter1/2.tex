Let's begin by defining what software architecture actually is. When you create an application, library, or any software component, you need to think about how the elements you write will look and how they will interact with each other. In other words, you're designing them and their relations with their surroundings. Just like with urban architecture, it's important to think about the bigger picture to not end up in a haphazard state. On a small scale, every single building looks okay, but they don't combine into a sensible bigger picture – they just don't fit together well. This is what's called accidental architecture and it is one of the outcomes you want to avoid. However, keep in mind that whether you're putting your thoughts into it or not, when writing software you are creating an architecture.


So, what exactly should you be creating if you want to mindfully define the architecture of your solution? The Software Engineering Institute has this to say:

\begin{center}
\tt
The software architecture of a system is the set of structures needed to reason about the system, which comprise software elements, relations among them, and properties of both.
\end{center}

This means that in order to define an architecture thoroughly, we should think about it from a few perspectives instead of just hopping into writing code.

\subsubsubsection{1.2.1\hspace{0.2cm}Different ways to look at architecture}

There are several scopes that can be used to look at architecture:

\begin{itemize}
\item 
Enterprise architecture deals with the whole company or even a group of companies. It takes a holistic approach and is concerned about the strategy of whole enterprises. When thinking about enterprise architecture, you should be looking at how all the systems in a company behave and cooperate with each other. It's concerned about the alignment between business and IT.

\item 
Solution architecture is less abstract than its enterprise counterpart. It stands somewhere in the middle between enterprise and software architecture. Usually, solution architecture is concerned with one specific system and the way it interacts with its surroundings. A solution architect needs to come up with a way to fulfill a specific business need, usually by designing a whole software system or modifying existing ones.

\item
Software architecture is even more concrete than solution architecture. It concentrates on a specific project, the technologies it uses, and how it interacts with other projects. A software architect is interested in the internals of the project's components.

\item
Infrastructure architecture is, as the name suggests, concerned about the infrastructure that the software will use. It defines the deployment environment and strategy, how the application will scale, failover handling, site reliability, and other infrastructure-oriented aspects.

\end{itemize}

Solution architecture is based on both software and infrastructure architectures to satisfy the business requirements. In the following sections, we will talk about both those aspects to prepare you for both small- and large-scale architecture design. Before we jump into that, let's also answer one fundamental question: why is architecture important?













