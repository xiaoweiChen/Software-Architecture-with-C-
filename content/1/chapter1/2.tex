先从定义软件架构开始。在创建应用程序、库或软件组件时,需要考虑所编写的元素的表现方式,以及交互方式。换句话说,在设计它们以及与周围环境的关系时,就需要像建设城市一样,要规划更大的蓝图,要从大局出发,不要陷入杂乱无章的状态。小范围内,每一栋建筑看起来都不错,但它们并不能组成一个更大的整体——只是不太适合,这就是所谓的\textit{随意架构},这需要避免。请记住,无论是否将自己的想法放入其中,在编写软件时,都是在构建一个架构。

如果想要谨慎地定义解决方案的架构,需要创建什么呢?软件工程学如是说:

\begin{center}
\tt
系统的软件架构是系统运行所需的组件,包括软件元素、元素间的关系,以及元素的属性。
\end{center}

这样,为了定义一个架构,需要从不同的角度来考虑,而非直接开始写代码。

\subsubsubsection{1.2.1\hspace{0.2cm}从不同的角度来看待架构}

这里提供几个思考的角度:

\begin{itemize}
\item 
企业架构涉及整个公司,甚至是集团公司。其采取整体的方法,只关注企业的战略。考虑企业架构时,应该了解公司中的所有系统的行为和相互合作的模式。其对业务和IT之间的一致性相当敏感。

\item 
解决方案架构不像企业架构那样抽象,它介于企业架构和软件架构之间。通常,解决方案架构与特定的系统,以及与周围环境交互的方式有关。解决方案架构师需要想出一种方法来满足特定的业务需求,通常的方法是通过设计整个软件系统或修改现有的系统。

\item
软件架构比解决方案架构更加具体,其集中于一个特定的项目,使用特定的技术,以及关注如何与其他项目进行交互。这里,软件架构师感兴趣的是项目组件的内部结构。

\item
基础设施架构,关注的是软件使用的基础设施。其定义了部署环境和策略、应用程序如何扩展、故障转移处理、站点可靠性以及其他面向基础设施的方面。
\end{itemize}

解决方案架构基于软件和基础设施架构,从而满足业务需求。在后续的内容平中,将对这两种架构进行讨论,以便之后为小型和大型的架构设计做准备。在开始之前,先提一个问题:为什么架构很重要?













