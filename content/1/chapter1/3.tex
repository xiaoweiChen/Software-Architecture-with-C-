
实际上,应该这样问:为什么要关心架构重要性?无论是否有意识地去构建,最终都会得到某种类型的架构。如果在几个月,甚至几年的开发之后,仍希望软件保证质量,需要在早期采取一些措施。如果不考虑架构,那么很可能永远不会呈现期望的品质。

因此,为了让产品满足业务需求和性能、可维护性、可扩展性(等),就需要关注其架构,最好尽早这样做。现在,来聊聊优秀架构师不能接受的两件事。

\subsubsubsection{1.3.1\hspace{0.2cm}软件腐烂}

完成起始的工作,并在脑中有了特定的架构后,还需要持续地控制架构后续的发展,以及是否符合其用户的需求,因为这些需求在软件的开发和生命周期中也可能发生变化。软件腐烂,有时也称为\textit{代码腐烂},发生在实现决策与计划架构不匹配的时候。所有这些差异都应视为\textit{技术债}。

\subsubsubsection{1.3.2\hspace{0.2cm}随意架构}

如果不能跟踪开发是否遵循所选择的架构,或者没有计划架构的样子,通常会导致随意架构。不管在其他领域应用最佳实践,例如:测试或具有特定的开发文化,随意架构都会发生。

有几个反模式可以用来验证随意架构。代码就像一个大泥球,这是最明显的。通常,如果软件是紧密耦合的,可能是循环依赖,但在一开始并不是这样,这就是一个重要的信号,表明应该更有意识地关注架构的外在表现。

现在,来了解一下架构师必须完成哪些工作,才能交付一个可行的解决方案。









