We've used the term DevOps (and DevSecOps) several times within this book. This topic deserves some additional space, in our opinion. DevOps is an approach to building software products that breaks with traditional silo-based development.

In the waterfall model, teams operated on single aspects of work independently of each other. The development team would write code, QA would test and validate the code, and security and compliance would come after that. Eventually, the operations team would take care of maintenance. The teams rarely communicated, and even then, it was usually a very formal process.

Knowledge about particular fields of expertise was only available to the teams responsible for a given piece of the workflow. Developers knew very little about QA and next to nothing about operations. While this setup was very convenient, the modern landscape requires more agility than the waterfall model can provide.

That's why a new model of working was proposed, one that encourages more collaboration, better communication, and lots of knowledge sharing between different stakeholders of a software product. While DevOps refers to bringing together developers and operations, what it means is bringing everyone closer.

Developers start working with QA and security even before they write the first lines of code. Operation engineers are more familiar with the code base. Businesses can easily track the progress of a given ticket and, in some cases, can even do a deployment preview in a self-service manner.

DevOps has become synonymous with using particular tools such as Terraform or Kubernetes. But DevOps is by no means the same as using any specific tools. Your organization can follow the DevOps principles without using Terraform or Kubernetes, and it can use Terraform and Kubernetes while not practicing DevOps.

One of the principles of DevOps is that it encourages improved information flow among the product's stakeholders. With that, it's possible to fulfill another principle: reduce wasteful activities that don't bring value to the end product.

When you're building modern systems, it is worth doing so using modern methodology. Migrating an existing organization to DevOps may require a massive mindset shift, so it is not always possible. It's worth pursuing when starting a greenfield project that you have control over.