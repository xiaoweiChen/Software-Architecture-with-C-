

这个问题的答案取决于几个因素。这里列出了一些:

\begin{itemize}
\item 
如果想要存储时间序列(以较小的、定期的间隔保存增量),那么最好的选择便是使用InfluxDB或victoria ametrics。

\item
如果需要类似SQL但不需要连接,或者换句话说,如果计划将数据存储在列中,可以尝试使用Apache Cassandra、AWS DynamoDB或Google的BigTable。

\item
如果不是这样,那么应该考虑数据是否是没有模式的文档,比如JSON或某种应用程序日志。如果是这种情况,可以使用Elasticsearch,它非常适合灵活的数据,并提供RESTful API。也可以试试MongoDB,它以二进制JSON (BSON)格式存储数据,并允许使用MapReduce。
\end{itemize}

好!但是如果不想存储文档呢?可以选择对象存储,尤其是在数据很大的情况下。通常,在这种情况下,与云提供商合作是不错的选择,这意味着使用Amazon的S3或Google的云存储,或Microsoft的Blob存储应该比较合适。如果想使用本地的工具,可以使用OpenStack的Swift或者部署Ceph。

如果文件存储也不是想要的,那么可能只是存储简单的键值数据,使用这种存储方式的好处是速度快。这就是为什么很多分布式缓存都使用它来构建的原因。值得注意的技术包括Riak、Redis和Memcached(最后一个不适合持久化数据)。

除了前面提到的选项之外,还可以考虑使用基于树的数据库,如BerkeleyDB。这些数据库基本上是特殊的键值存储,具有类似路径的访问。如果树对目前的需求限制太多,可能会对面向图形的数据库感兴趣,如Neo4j或OrientDB。







