

The answer to this question depends on several factors. A few are listed here:


\begin{itemize}
\item 
If you want to store time series (save increments at small, regular intervals), then the best option would be to use InfluxDB or VictoriaMetrics.

\item
If you need something similar to SQL but could live without joins, or in other words, if you plan to store your data in columns, you can try out Apache Cassandra, AWS DynamoDB, or Google's BigTable.

\item
If that's not the case, then you should think about whether your data is a document without a schema, such as JSON or some kind of application logs. If that's the case, you could go with Elasticsearch, which is great for such flexible data and provides a RESTful API. You could also try out MongoDB, which stores its data in Binary JSON (BSON) format and allows MapReduce.
\end{itemize}

OK, but what if you don't want to store documents? Then you could opt for object storage, especially if your data is large. Usually, going with a cloud provider is OK in this case, which means that using Amazon's S3 or Google's Cloud Storage or Microsoft's Blob storage should help your case. If you want to go with something local, you could use OpenStack's Swift or deploy Ceph.

If file storage is also not what you're looking for, then perhaps your case is just about simple key-value data. Using such storage has its benefits as it's fast. This is why many distributed caches are built using it. Notable technologies include Riak, Redis, and Memcached (this last one is not suitable for persisting data).

Aside from the previously mentioned options, you could consider using a tree-based database such as BerkeleyDB. Those databases are basically specialized key-value storage with path-like access. If trees are too restricting for your case, you might be interested in graph-oriented databases such as Neo4j or OrientDB.







