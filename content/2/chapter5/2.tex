
尽管C++允许使用面向对象API(如果使用基于coffee的语言编写代码,可能会熟悉这些API),但它还有其他一些诀窍,将在本节中提到。

\subsubsubsection{5.2.1\hspace{0.2cm}使用RAII}

C API和C++ API的主要区别是什么?这与多态或类本身无关,而是与RAII用法有关。

\textbf{RAII}表示\textbf{资源获取即初始化},实际上更多地是关于释放资源,而不是获取资源。来看一下用C和C++编写的类似API,来展示这个特性的实际应用情况:

\begin{lstlisting}[style=styleCXX]
struct Resource;

// C API
Resource* acquireResource();
void releaseResource(Resource *resource);

// C++ API
using ResourceRaii = std::unique_ptr<Resource, decltype(&releaseResource)>;
ResourceRaii acquireResourceRaii();
\end{lstlisting}

C++ API是基于C API的,但并不总是如此。在C++ API中,不需要单独的函数来释放资源。由于RAII用法,当ResourceRaii对象超出作用域,就会自动释放。这样省去了手工管理资源的负担,而且不需要额外的成本。

更重要的是,不需要编写类——可以重用标准库的\textit{unique\_ptr},这是一个轻量级指针。确保所管理的对象总可以释放,并且只释放一次。

因为要管理一些特殊类型的资源,所以必须使用自定义删除类型。\textit{acquireResourceRaii}函数需要将实际的指针传递给\textit{releaseResource}函数。如果只想在C++中使用C API,则不需要向用户公开。

这里需要注意的事情是,RAII不仅仅用于管理内存:可以使用它处理任何资源的所有权,例如锁、文件句柄、数据库连接,以及在其RAII包装器超出作用域时应该释放的所有东西。

\subsubsubsection{5.2.2\hspace{0.2cm}整理C++中容器的接口}

标准库实现是惯用的、性能好的C++代码。如果想阅读一些有趣的模板代码,应该去\textit{std::chrono}看一看,其演示了一些有用的技术,并用新的方法来进行实现。在\textit{扩展阅读}部分可以找到有关libstdc++实现的链接。

当涉及到库的其他位置时,快速浏览一下它的容器,会发现其接口与其他编程语言中的有所不同。为了说明这一点,来看一下来自标准库的一个简单的类,\textit{std::array},并对其进行分析:

\begin{lstlisting}[style=styleCXX]
template <class T, size_t N>
struct array {
	// types:
	typedef T& reference;
	typedef const T& const_reference;
	typedef /*implementation-defined*/ iterator;
	typedef /*implementation-defined*/ const_iterator;
	typedef size_t size_type;
	typedef ptrdiff_t difference_type;
	typedef T value_type;
	typedef T* pointer;
	typedef const T* const_pointer;
	typedef reverse_iterator<iterator> reverse_iterator;
	typedef reverse_iterator<const_iterator> const_reverse_iterator;
\end{lstlisting}

开始阅读类定义时,首先看到的是它为某些类型创建了别名。这在标准容器中很常见,并且在许多容器中别名的名称是相同的。这种情况的出现有几个原因,其中之一是——这样做可以节省开发人员花在挠头和试图理解创建者的意思,以及特定别名是如何命名的时间。另一个原因是类和库编写器的用户在编写自己的代码时,常常依赖于此类类型特征。如果容器不提供这样的别名,将使它与一些标准实用程序或类型特征一起使用变得更加困难,因此使用API的用户需要解决这个问题,甚至使用一个完全不同的类。

即使没有在模板中使用这些类型别名,也可以使用它们。函数参数和类成员字段依赖于这些类型并不罕见,如果正在编写一个其他人可以使用的类,请始终记住提供这些类型。例如,正在编写一个分配器,许多使用者将依赖于特定类型别名的存在。

来看看array类会带来什么:

\begin{lstlisting}[style=styleCXX]
// no explicit construct/copy/destroy for aggregate type
\end{lstlisting}

\textit{std::array}没有构造函数的定义,包括复制/移动构造函数;赋值运算符;或析构函数。通常,在不必要的时候添加这样的成员实际上会损害性能。使用非默认构造函数(和\textit{T() \{\}}是非默认的,而不是\textit{T() = default;}),类不再简单,也不再有简单的构造函数,这将阻止编译器对其进行优化。

看看类还有哪些声明:

\begin{lstlisting}[style=styleCXX]
constexpr void fill(const T& u);
constexpr void swap(array<T, N>&) noexcept(is_nothrow_swappable_v<T&>);
\end{lstlisting}

可以看到两个成员函数,包括一个成员交换。通常,不依赖于\textit{std::swap}的默认行为,提供我们自己的行为是有益的,例如在\textit{std::vector}中,底层存储作为一个整体进行交换,而不是交换每个元素。当写一个成员\textit{swap}函数时,一定要引入名为\textit{swap}的自由函数,这样它就可以通过\textbf{参数依赖查找(ADL)}检测到,从而可以调用成员函数\textit{swap}。

关于\textit{swap}函数,是有条件的\textit{noexcept}。如果存储的类型可以在不引发异常的情况下进行交换,则数组的交换也将是\textit{noexcept}。使用非异常抛出交换可以将类型作为成员存储的类的复制操作中呈现强异常安全的保证。

如下面的代码块所示,现在出现了一组大的函数,展示了许多类的另一个重要组件——迭代器:

\begin{lstlisting}[style=styleCXX]
	 // iterators:
	constexpr iterator begin() noexcept;
	constexpr const_iterator begin() const noexcept;
	constexpr iterator end() noexcept;
	constexpr const_iterator end() const noexcept;
	
	constexpr reverse_iterator rbegin() noexcept;
	constexpr const_reverse_iterator rbegin() const noexcept;
	constexpr reverse_iterator rend() noexcept;
	constexpr const_reverse_iterator rend() const noexcept;
	
	constexpr const_iterator cbegin() const noexcept;
	constexpr const_iterator cend() const noexcept;
	constexpr const_reverse_iterator crbegin() const noexcept;
	constexpr const_reverse_iterator crend() const noexcept;
\end{lstlisting}

迭代器对于每个容器都至关重要。如果不为类提供迭代器访问,就不能在基于范围的for循环中使用,而且也不能与标准库中所有有用的算法兼容。这并不意味着需要编写自己的迭代器类型——如果存储是连续的,只需使用简单的指针即可。提供\textit{const}迭代器可以以不可变的方式使用类,而提供反向迭代器可以为容器提供更多用例。

继续往下看:

\begin{lstlisting}[style=styleCXX]
	// capacity:
	constexpr size_type size() const noexcept;
	constexpr size_type max_size() const noexcept;
	constexpr bool empty() const noexcept;
	
	// element access:
	constexpr reference operator[](size_type n);
	constexpr const_reference operator[](size_type n) const;
	constexpr const_reference at(size_type n) const;
	constexpr reference at(size_type n);
	constexpr reference front();
	constexpr const_reference front() const;
	constexpr reference back();
	constexpr const_reference back() const;
	constexpr T * data() noexcept;
	constexpr const T * data() const noexcept;
private:
	// the actual storage, like T elements[N];
};
\end{lstlisting}

迭代器之后,有几种方法来检查和修改容器的数据。\textit{array}中,它们是\textit{constexpr}。这意味着,如果要编写编译时代码,可以使用数组类。我们将在本章后面的\textit{编译时计算}部分更多的进行了解。

最后,完成了\textit{array}的定义。然而,从C++17开始,在类型定义之后,会发现如下代码:

\begin{lstlisting}[style=styleCXX]
template<class T, class... U>
	array(T, U...) -> array<T, 1 + sizeof...(U)>;
\end{lstlisting}

这样的语句被称为\textbf{演绎指南},是称为\textbf{类模板参数演绎(CTAD)}特性的一部分,这个特性在C++17中引入,允许在声明变量时忽略模板参数。其有利于\textit{array},现在可以只写以下内容:

\begin{lstlisting}[style=styleCXX]
auto ints = std::array{1, 2, 3};
\end{lstlisting}

然而,对于更复杂的类型,例如map,可能使用起来更方便:

\begin{lstlisting}[style=styleCXX]
auto legCount = std::unordered_map{ std::pair{"cat", 4}, {"human", 2}, {"mushroom", 1} };
\end{lstlisting}

然而,这里有一个catch:当传递第一个参数时,需要指定传递的是键值对(注意,还使用了一个推导指引)。

既然正在讨论接口的主题,就再看看接口的其他方面。

\subsubsubsection{5.2.3\hspace{0.2cm}接口中使用指针}

接口中使用的类型非常重要。即使有文档,好的API也应该是一目了然的。看看向函数传递资源参数的不同方法,如何向API使用者提出不同的建议。

考虑下面的函数声明:

\begin{lstlisting}[style=styleCXX]
void A(Resource*);
void B(Resource&);
void C(std::unique_ptr<Resource>);
void D(std::unique_ptr<Resource>&);
void E(std::shared_ptr<Resource>);
void F(std::shared_ptr<Resource>&);
\end{lstlisting}

应该什么时候使用这些函数中的哪一个?

由于智能指针现在是处理资源的标准方式,\textit{A}和\textit{B}应该留给简单的参数传递,如果不处理传递对象的所有权,就不应该使用。A应该只用于单个资源,想传递多个实例,可以使用一个容器,例如\textit{std::span}。如果确定想传递的对象不为空,最好使用引用来传递它,比如const引用。如果对象不是太大,也可以考虑按值传递。

关于函数\textit{C}到\textit{F}的一个很好的经验法则是,如果想操作指针本身,应该只将智能指针作为参数传递;例如,用于转移所有权。

\textit{C}函数根据值接受\textit{unique\_ptr},这意味着它是一个资源池。换句话说,它先消耗资源,然后释放资源。请注意,仅仅通过选择特定的类型,接口就可以清楚地表达其意图。

只有想要传入包含资源的\textit{unique\_ptr},并接收另一个包含相同\textit{unique\_ptr}的资源作为输出参数时,才应该使用\textit{D}函数。这并不是一个好主意,因为它要求调用者将其存储在一个\textit{unique\_ptr}中。换句话说,如果想传递一个const \textit{unique\_ptr<Resource>\&},换成传递一个\textit{Resource*}(或\textit{Resource\&})会更好。

\textit{E}函数用于与被调用方共享资源所有权。通过值传递\textit{shared\_ptr}的开销相对比较大,因为需要增加它的引用计数器。然而,通过值传递\textit{shared\_ptr}是可以的,如果被调用方真的想成为共享的所有者,就必须在某处创建一个副本。

\textit{F}函数类似于\textit{D},应该只在想要操作\textit{shared\_ptr}实例并通过这个in/out参数传播更改时使用。如果不确定函数是否应该拥有所有权,可以考虑传递\textit{const shared\_ptr\&}。

\subsubsubsection{5.2.4\hspace{0.2cm}确定前置条件和后置条件}

函数对参数有一些要求很正常,每一项要求都应作为先决条件。如果函数可以保证结果具有某些属性——例如,是非负的——该函数也应该明确这一点。一些开发人员使用注释来告知其他人,但并没有真正执行需求。使用if语句更好,但隐藏了检查的原因。目前,C++标准仍然没有提供一种方法来处理这个问题(契约第一次加入是在C++20标准,只是后面删除了)。幸运的是,像Microsoft的\textbf{指南支持库(GSL)}这样的库提供了自己实现的检查。

假设,出于某种原因,正在编写自己的队列。push成员函数看起来像这样:

\begin{lstlisting}[style=styleCXX]
template<typename T>
T& Queue::push(T&& val) {
	gsl::Expects(!this->full());
	// push the element
	gsl::Ensures(!this->empty());
}
\end{lstlisting}

注意,用户不需要访问实现来确保某些检查已经就绪。代码也是自解释的,因为函数需要什么,结果会是什么都很清楚。

\subsubsubsection{5.2.5\hspace{0.2cm}使用内联名称空间}

系统编程中,通常并不总是针对API编写代码,还需要关心ABI兼容性。一个著名的ABI崩溃发生在GCC发布第5个版本的时候,其中一个主要的变化是改变了\textit{std::string}的布局。这意味着与旧的GCC版本一起工作的库(或者仍然在新版本中使用新的ABI,这在最近的GCC版本中仍然是一种现象)不能与使用新版本ABI编写的代码一起工作。在ABI不同的情况下,如果收到一个链接器错误,可以认为自己很幸运。某些情况下,例如混合\textit{NDEBUG}代码和\textit{DEBUG}代码。若类只有一个这样的配置中可用的成员,为了更好地调试而添加的成员,可能会出现内存崩溃。

使用C++11的内联名称空间时,一些通常很难调试的内存损坏很容易转化为链接器错误。考虑下面的代码:

\begin{lstlisting}[style=styleCXX]
#ifdef NDEBUG
inline namespace release {
#else
inline namespace debug {
#endif

struct EasilyDebuggable {
// ...
#ifndef NDEBUG
// fields helping with debugging
#endif
};

} // end namespace
\end{lstlisting}

由于前面的代码使用内联名称空间,当声明这个类的对象时,用户不会看到这两种构建类型之间的区别:来自内联名称空间的所有声明在周围的作用域中都是可见的。但是,链接器将以不同的符号名结束,这将导致链接器在试图链接不兼容的库时失败,从而给提供了我们正在寻找的ABI安全性,并提供了一个很好的错误消息,其中提到了内联名称空间。

有关提供安全优雅ABI的更多信息,请参见Arvid Norberg的《C++ Now 2019》ABI挑战演讲,链接在扩展阅读部分。

\subsubsubsection{5.2.6\hspace{0.2cm}使用\texttt{std::optional}}

从ABI回到API,再提一种在本书前面设计优秀API时忽略的类型。当涉及到函数的可选参数时,本节的主角可以挽救一切,因为它可以帮助类型拥有可能有值也可能没有值的组件,它还可以用于设计干净的接口或替代指针。它就是\texttt{std::optional},在C++17中标准化了。如果不能使用C++17,可以在Abseil(\texttt{absl::optional})中找到它,或者从Boost(\texttt{Boost::optional})中找到非常类似的类型。使用该类的一个很大的优点是,非常清楚地表达了意图,这有助于编写干净和自文档化的接口。来看看实际情况。

\hspace*{\fill} \\ %插入空行
\noindent
\textbf{可选的函数参数}

首先将参数传递给可以(但可能不能)保存值的函数。曾经发现过类似于以下的函数签名吗?

\begin{lstlisting}[style=styleCXX]
void calculate(int param); // If param equals -1 it means "no value"
void calculate(int param = -1);
\end{lstlisting}

有时,如果\texttt{param}是在代码的其他地方计算的,那么很容易错误地传递-1——可能它是一个有效值。下面的签名呢?

\begin{lstlisting}[style=styleCXX]
void calculate(std::optional<int> param);
\end{lstlisting}

这一次,如果不想传递一个值,该做什么就更清楚了:只传递一个空的可选值。目的很明确,-1仍然可以用作有效值,而不必以一种类型不安全的方式赋予它特殊的含义。

这只是optional模板的一种用法。

\hspace*{\fill} \\ %插入空行
\noindent
\textbf{optional函数返回值}

就像接受特殊值表示形参空值一样,函数有时也会空值返回。看看更喜欢下面哪一种?

\begin{lstlisting}[style=styleCXX]
int try_parse(std::string_view maybe_number);
bool try_parse(std::string_view maybe_number, int &parsed_number);
int *try_parse(std::string_view maybe_number);
std::optional<int> try_parse(std::string_view maybe_number);
\end{lstlisting}

如何知道第一个函数在出现错误时将返回什么值?或者会抛出一个异常,而不是返回一个魔值吗?继续看第二个签名,如果有错误,看起来会返回\texttt{false},但很容易忘记检查,并直接读取\texttt{parsed\_number}。第三种情况下,虽然假设错误时返回\texttt{nullptr},成功时返回一个整数相对安全,但现在还不清楚是否应该释放返回的整数。

对于最后一个签名,只要查看它,就可以清楚地看到在发生错误时将返回一个空值,并且不需要做任何操作。就是这么简单、易懂、优雅!

optional的返回值也可以用来标记没有返回值,而不一定表示发生了错误。话虽如此,来看看optional的最后一个用例。

\hspace*{\fill} \\ %插入空行
\noindent
\textbf{optional类的成员}

类状态中实现一致性并不总是一件容易的任务,有时希望有一两个成员不能简单地设置。可以使用可选的类成员,而不是为这种情况创建另一个类(这会增加代码的复杂性)或保留一个特殊的值(很容易不被注意地传递)。考虑以下类型:

\begin{lstlisting}[style=styleCXX]
struct UserProfile {
	std::string nickname;
	std::optional <std::string> full_name;
	std::optional <std::string> address;
	std::optional <PhoneNumber> phone;
};
\end{lstlisting}

可以看到哪些字段是必要的,哪些字段不需要填充。同样的数据可以用空字符串存储,但仅从结构体的定义来看并不明显。另一种选择是使用\textit{std::unique\_ptr},但是这样就会丢失局部性数据,而这对于性能来说是必不可少的。对于这种情况,\textit{std::optional}就非常合适。想要设计干净直观的API时,它绝对应该是开发者工具箱的一部分。

这些知识可以提供高质量和直观的API。还可以继续改进它们,这也会使编写的代码出现更少的错误。我们将在下一节对此进行讨论。










