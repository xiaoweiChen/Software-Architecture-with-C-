
尽管名称中有模式,模板递归模式(CRTP)是C++中的一种习惯用法。可以用来实现其他的习惯用法和设计模式,并应用静态多态性。

\subsubsubsection{6.3.1\hspace{0.2cm}何时使用动态多态性,何时使用静态多态性}

提到多态性,许多程序员会想到动态多态性,即在运行时收集执行函数调用所需的信息。与此相反,静态多态性是关于在编译时确定调用。前者的一个优点是可以在运行时修改类型列表,允许通过插件和库扩展类的层次结构。第二种方法的最大优点是,若事先知道类型,可以获得非常好的性能。当然,第一种情况下,有时可以期望编译器对调用进行反虚拟化,但不能总是期望它会这样做。但在第二种情况下,编译时间可能会更长。

不可能在所有情况下达到双赢。尽管如此,为类型选择正确的多态性类型仍有很长的路要走。如果性能受到威胁,强烈建议考虑静态多态性。CRTP是一个使用静态多态性的习惯用法。

许多设计模式都可以以各种方式实现。由于动态多态性的代价并不总合适,四人帮设计模式通常不是C++中的最佳解决方案。如果类型层次结构需要在运行时进行扩展,或者编译时间是一个比性能重要的问题(并且不打算很快使用模块),那么四人帮模式的经典实现可能是一个很好的选择。否则,可以尝试使用静态多态或以更简单的C++为中心的解决方案来实现,其中一些将在本章中描述。关键在于选择最适合的工具完成这项工作。

\subsubsubsection{6.3.2\hspace{0.2cm}实现静态多态性}

现在来实现静态多态类层次结构。目前需要一个基模板类:

\begin{lstlisting}[style=styleCXX]
template <typename ConcreteItem> class GlamorousItem {
	public:
	 void appear_in_full_glory() {
		static_cast<ConcreteItem *>(this)->appear_in_full_glory();
	}
};
\end{lstlisting}

基类的模板形参是派生类。这可能看起来很奇怪,但它可以静态地将接口函数转换为正确的类型。在本例中,命名为\texttt{appear\_in\_full\_glory}。然后在派生类中调用该函数的实现。派生类可以像这样实现:

\begin{lstlisting}[style=styleCXX]
class PinkHeels : public GlamorousItem<PinkHeels> {
	public:
	void appear_in_full_glory() {
		std::cout << "Pink high heels suddenly appeared in all their beauty\n";
	}
};

class GoldenWatch : public GlamorousItem<GoldenWatch> {
	public:
	void appear_in_full_glory() {
		std::cout << "Everyone wanted to watch this watch\n";
	}
};
\end{lstlisting}

这些类都使用自身作为模板参数派生自基类\texttt{GlamorousItem}。它们还实现了所需的功能。

请注意,与动态多态性不同,CRTP中的基类是一个模板,因此每个派生类将获得不同的基类型。这意味着很难创建GlamorousItem基类的容器。不过,还可以做以下几件事:

\begin{itemize}
\item 
存储在元组中。

\item 
创建派生类的\texttt{std::variant}。

\item 
添加一个公共类来包装Base的所有实例化,也可以为这个版本使用变体。
\end{itemize}

第一种情况下,可以像下面这样使用这个类。首先,创建基类的实例元组:

\begin{lstlisting}[style=styleCXX]
template <typename... Args>
using PreciousItems = std::tuple<GlamorousItem<Args>...>;

auto glamorous_items = PreciousItems<PinkHeels, GoldenWatch>{};
\end{lstlisting}

类型别名元组将能够存储GlamorousItem。现在,需要做的就是调用这个函数:

\begin{lstlisting}[style=styleCXX]
std::apply(
	[]<typename... T>(GlamorousItem<T>... items) {
		(items.appear_in_full_glory(), ...); },
	glamorous_items);
\end{lstlisting}

这里试图迭代一个元组,所以最简单的方法是使用\texttt{std::apply},在给定元组的元素上调用给定的可调用对象。在例子中,可调用对象是一个只接受GlamorousItem基类的lambda。这里使用了C++17中引入的折叠表达式,以确保所有元素都能调用这个函数。

如果需要使用一个变量来代替元组,可以使用\texttt{std::visit},就像这样:

\begin{lstlisting}[style=styleCXX]
	using GlamorousVariant = std::variant<PinkHeels, GoldenWatch>;
	auto glamorous_items = std::array{GlamorousVariant{PinkHeels{}},
		GlamorousVariant{GoldenWatch{}}};
	for (auto& elem : glamorous_items) {
		std::visit([]<typename T>(GlamorousItem<T> item){
			item.appear_in_full_glory(); }, elem);
}
\end{lstlisting}

\texttt{std::visit}函数基本上接受\texttt{std::variant},并在其中存储的对象上调用传入的lambda。这里,创建了一个\texttt{GlamorousVariant}数组,因此可以像对其他容器一样对其进行迭代,使用适当的lambda访问每个变量。

如果从用户使用接口的角度编写不是很直观,考虑下一个方法,将\texttt{variant}包装到另一个类中,在例子中称为\texttt{CommonGlamorousItem}:

\begin{lstlisting}[style=styleCXX]
class CommonGlamorousItem {
public:
	template <typename T> requires std::is_base_of_v<GlamorousItem<T>, T>
	explicit CommonGlamorousItem(T &&item)
		: item_{std::forward<T>(item)} {}
private:
	GlamorousVariant item_;
};
\end{lstlisting}

为了构造包装器,使用转发构造函数(模板化的\texttt{T\&\&}是它的参数)。然后向前移动,而不是移动创建\texttt{item\_ wrapped}的\texttt{variant},因为这样只移动右值输入。这里还要约束模板参数,因此一方面需要包装\texttt{GlamorousItem}基类,另一方面模板不可用作移动或复制构造函数。

还需要包装成员函数:

\begin{lstlisting}[style=styleCXX]
void appear_in_full_glory() {
	std::visit(
	[]<typename T>(GlamorousItem<T> item) {
		item.appear_in_full_glory(); },
	item_);
}
\end{lstlisting}

这次,\texttt{std::visit}的使用是一个实现的技巧。用户可以按以下方式使用这个包装器类:

\begin{lstlisting}[style=styleCXX]
auto glamorous_items = std::array{CommonGlamorousItem{PinkHeels{}},
		   CommonGlamorousItem{GoldenWatch{}}};
	for (auto& elem : glamorous_items) {
		elem.appear_in_full_glory();
	}
\end{lstlisting}

这种方法可以让类的用户编写易于理解的代码,但仍然保持静态多态性的性能。

为了提供类似的用户体验,尽管性能较差,还可以使用一种称为类型擦除的技术,将在下文中讨论。

\subsubsubsection{6.3.3\hspace{0.2cm}使用类型擦除}

尽管类型擦除与CRTP无关,但它与当前的示例非常适配,这也是在这里展示它的原因。

类型擦除习惯用法是关于在多态接口下隐藏具体类型。Sean Parent在GoingNative 2013大会上的演讲《Inheritance Is The Base Class of Evil》就是一个很好的例子。强烈推荐在业余时间观看,可以在扩展阅读部分找到链接。在标准库中,可以在\texttt{std::function},\texttt{std::shared\_ptr}的删除器或\texttt{std::any}中找到它。

方便性和灵活性是有代价的——需要使用指针和虚拟调度,这使得标准库中提到的实用程序在面向性能的用例中不适合使用。要多加小心。

为了在示例中引入类型擦除,不再需要CRTP。这一次,\texttt{GlamorousItem}类将把动态多态对象包装在一个智能指针中:

\begin{lstlisting}[style=styleCXX]
class GlamorousItem {
public:
	template <typename T>
	explicit GlamorousItem(T t)
	: item_{std::make_unique<TypeErasedItem<T>>(std::move(t))} {}
	
	void appear_in_full_glory() { item_->appear_in_full_glory_impl(); }
	
private:
	std::unique_ptr<TypeErasedItemBase> item_;
};
\end{lstlisting}

这次,存储一个指向基类(\texttt{TypeErasedItemBase})的指针,指向派生包装(\texttt{TypeErasedItem<T>})。基类可以定义如下:

\begin{lstlisting}[style=styleCXX]
struct TypeErasedItemBase {
	virtual ~TypeErasedItemBase() = default;
	virtual void appear_in_full_glory_impl() = 0;
};
\end{lstlisting}

每个派生包装器也需要实现这个接口:

\begin{lstlisting}[style=styleCXX]
template <typename T> class TypeErasedItem final : public
TypeErasedItemBase {
public:
	explicit TypeErasedItem(T t) : t_{std::move(t)} {}
	void appear_in_full_glory_impl() override { t_.appear_in_full_glory();
	}

private:
	T t_;
};
\end{lstlisting}

基类的接口是通过从包装对象调用函数来实现的。注意,这个习惯称为“类型擦除”,因为\texttt{GlamorousItem}类并不知道实际包装的是什么\texttt{T}。信息类型在项构造时删除,但它仍然工作,因为\texttt{T}实现了所需的方法。

具体项目可以以更简单的方式实现,如下所示:

\begin{lstlisting}[style=styleCXX]
class PinkHeels {
	public:
	void appear_in_full_glory() {
		std::cout << "Pink high heels suddenly appeared in all their beauty\n";
    }
};

class GoldenWatch {
public:
	void appear_in_full_glory() {
		std::cout << "Everyone wanted to watch this watch\n";
	}
};
\end{lstlisting}

这次,不需要从任何基类继承。需要的只是鸭子类型——如果它叫得像鸭子,那很可能就是鸭子。

类型擦除API可以这样使用:

\begin{lstlisting}[style=styleCXX]
auto glamorous_items =
	std::array{GlamorousItem{PinkHeels{}}, GlamorousItem{GoldenWatch{}}};
for (auto &item : glamorous_items) {
	item.appear_in_full_glory();
}
\end{lstlisting}

只创建一个包装器数组并对其进行迭代,所有这些都使用基于值的语义。这里使用它是令人愉快的,因为多态性作为实现细节对调用者是隐藏的。

然而,正如前面提到的,这种方法的一个大缺点是性能差。类型擦除是有代价的,所以应该谨慎使用,绝对不要在热路径中使用。

既然已经描述了如何包装和删除类型,继续来讨论如何创建类型。

















