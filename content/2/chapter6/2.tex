
如果熟悉面向对象编程语言,那么一定听说过“四人帮”的设计模式。虽然可以用C++实现(通常也是这样),但这种多范例语言通常采用不同的方法来实现相同的目标。如果想要击败所谓的基于coffe的语言(如Java或C\#)的性能,有时虚拟调度的代价太大了。很多情况下,一开始就知道要处理什么类型的问题。如果是这种情况,可以使用该语言和标准库中提供的工具编写更多性能更好的代码。在众多习语中,先从了解语言习语开始。开始我们的旅程吧!

根据定义,习语是在给定语言中反复出现的结构,是特定于该语言的表达式。以C++为\textit{母语}的人应该凭直觉知道它的习语。之前已经提到了智能指针,这是最常见的一种。现在来讨论一个类似的例子。

\subsubsubsection{6.2.1\hspace{0.2cm}使用RAII保护自动化范围的退出操作}

C++中最强大的表达式之一是关闭作用域的大括号。这是调用析构函数和产生RAII魔法的地方。要驯服这个咒语,不需要使用智能指针。所需要的是一个RAII守卫——一个对象,构建,记住需要做什么,销毁。这样,无论作用域是正常存在还是异常存在,工作都将自动完成。

最好的部分——甚至不需要从头编写一个RAII保护程序,各种库中已经存在经过良好测试的实现。如果正在使用GSL,可以使用\texttt{GSL::finally()}。考虑下面的例子:

\begin{lstlisting}[style=styleCXX]
using namespace std::chrono;
void self_measuring_function() {
	auto timestamp_begin = high_resolution_clock::now();
	
	auto cleanup = gsl::finally([timestamp_begin] {
		auto timestamp_end = high_resolution_clock::now();
		std::cout << "Execution took: " <<
		duration_cast<microseconds>(timestamp_end - timestamp_begin).count() << "
		us";
	});

	// perform work
	// throw std::runtime_error{"Unexpected fault"};
}
\end{lstlisting}

这里,我们在函数的开始处使用一个时间戳,在函数的结束处使用另一个时间戳。试着运行这个例子,看看取消注释\texttt{throw}语句会如何影响执行。这两种情况下,RAII守卫将正确地打印执行时间(假设异常在某个地方捕获)。

现在让我们讨论一些更流行的C++习惯用法。

\subsubsubsection{6.2.2\hspace{0.2cm}管理可复制性和可移动性}

C++中设计新类型时,确定它是否可复制和可移动很重要,更重要的是正确实现类的语义。现在来讨论一下这些问题。

\hspace*{\fill} \\ %插入空行
\noindent
\textbf{实现不可复制的类型}

某些情况下,不希望类被复制,复制代价非常昂贵的类就是一个例子。另一种可能是由于切片而导致的错误。过去,防止此类对象复制的一种常见方法是使用\texttt{noncopyable}习惯用法:

\begin{lstlisting}[style=styleCXX]
struct Noncopyable {
	Noncopyable() = default;
	Noncopyable(const Noncopyable&) = delete;
	Noncopyable& operator=(const Noncopyable&) = delete;
};

class MyType : NonCopyable {};
\end{lstlisting}

但注意,这样的类也是不可移动的,尽管在阅读类定义时很容易忽略。更好的方法是显式地添加两个缺失的成员(移动构造函数和移动赋值操作符)。作为经验法则,当声明这些特殊成员函数时,总是声明所有的函数。这意味着从C++11开始,首选的方式将是编写以下代码:

\begin{lstlisting}[style=styleCXX]
struct MyTypeV2 {
	MyTypeV2() = default;
	MyTypeV2(const MyTypeV2 &) = delete;
	MyTypeV2 & operator=(const MyTypeV2 &) = delete;
	MyTypeV2(MyTypeV2 &&) = delete;
	MyTypeV2 & operator=(MyTypeV2 &&) = delete;
};
\end{lstlisting}

这次,成员是直接在目标类型中定义的,没有使用帮助性质的\texttt{NonCopyable}类型。

\hspace*{\fill} \\ %插入空行
\noindent
\textbf{遵守三五规则}

讨论特殊成员函数时,还有一件事需要提到:如果不删除它们,而是提供自己的实现,那么很可能需要定义所有的成员函数,包括析构函数。这在C++98中称为3规则(因为需要定义三个函数:复制构造函数、复制赋值操作符和析构函数),而由于C++11的移动操作,现在它被5规则(另外两个是移动构造函数和移动赋值操作符)所取代。应用这些规则可以减少资源管理问题。

\hspace*{\fill} \\ %插入空行
\noindent
\textbf{坚持零原则}

如果只需要使用所有特殊成员函数的默认实现,那么就完全不要声明它们。这是一个明确的信号,表明希望使用默认行为,它也是最不容易混淆的。考虑以下类型:

\begin{lstlisting}[style=styleCXX]
class PotentiallyMisleading {
public:
	PotentiallyMisleading() = default;
	PotentiallyMisleading(const PotentiallyMisleading &) = default;
	PotentiallyMisleading &operator=(const PotentiallyMisleading &) =
default;
	PotentiallyMisleading(PotentiallyMisleading &&) = default;
	PotentiallyMisleading &operator=(PotentiallyMisleading &&) = default;
	~PotentiallyMisleading() = default;
	
private:
	std::unique_ptr<int> int_;
};
\end{lstlisting}

即使默认了所有成员,该类仍然是不可复制的。因为它有一个不可复制的\texttt{unique\_ptr}成员。幸运的是,Clang会就此进行警告,但是GCC在默认情况下不会这样做。更好的方法是应用零规则:

\begin{lstlisting}[style=styleCXX]
class RuleOfZero {
	std::unique_ptr<int> int_;
};
\end{lstlisting}

现在我们有更少的样板代码,通过查看成员,很容易注意到它不支持复制。

说到复制,还有一个更重要的习语需要了解。在此之前,将讨论另一个习惯用法,可以(也应该)用于实现第一个习惯用法。

\subsubsubsection{6.2.3\hspace{0.2cm}隐藏友元}

本质上,隐藏友元是在声明它们为友元的类型体中定义的非成员函数。这使得除了使用依赖于参数的查找(ADL)之外,无法以其他方式调用这类函数,从而有效地进行隐藏。因为减少了编译器考虑的重载数量,所以可以加快编译速度。这样做的好处是,它们提供的错误消息比其他选项更短。它们最后一个有趣的属性是,如果先进行隐式转换,则不能调用它们。这可以帮助您避免这种意外的转换。

虽然C++中的好友通常不推荐使用,但对于隐藏好友,情况会有所不同。如果上一段的优点不足够,那么还应该知道,它们应该是实现定制点的首选方式。现在,可能想知道定制点是什么。简单地说,是库代码使用的可调用对象,用户可以专门化它们的类型。标准库为这些对象保留了相当多的名称,例如\texttt{begin}、\texttt{end}、\texttt{反向变量}和\texttt{const变量}、\texttt{swap}、\texttt{(s)size}、\texttt{(c)data}和许多操作符等。如果决定为这些定制点中的一个提供自己的实现,那么其行为最好符合标准库的预期。

好了,理论到此为止。来看看如何在实践中使用隐藏友元来提供定制点特化。例如,创建一个简化的类来管理类型数组:

\begin{lstlisting}[style=styleCXX]
template <typename T> class Array {
public:
	Array(T *array, int size) : array_{array}, size_{size} {}
	
	~Array() { delete[] array_; }
	
	T &operator[](int index) { return array_[index]; }
	int size() const { return size_; }
	
	friend void swap(Array &left, Array &right) noexcept {
		using std::swap;
		swap(left.array_, right.array_);
		swap(left.size_, right.size_);
	}

private:
	T *array_;
	int size_;
};
\end{lstlisting}

这里定义了析构函数,这意味着还应该提供其他特殊的成员函数。这将在下一节使用隐藏友元交换实现。请注意,尽管在\texttt{Array}类的主体中声明了这个\texttt{swap}函数,但它仍然是一个非成员函数。它接受两个\texttt{Array}实例,但没有访问权限。

使用\texttt{std::swap}使编译器首先在交换成员的名称空间中查找交换函数,如果没有找到,它将返回到\texttt{std::swap}。这称为两步ADL和回退习惯用法,简称为两步,因为我们首先使\texttt{std::swap}可见,然后调用\texttt{swap}。\texttt{noexcept}关键字告诉编译器\texttt{swap}函数不会抛出异常,这使得它在某些情况下可以更快地生成代码。除了\texttt{swap}之外,出于同样的原因,也要用这个关键字标记默认构造函数和移动构造函数。

现在有了一个交换函数,可以用它对\texttt{Array}类应用另一个习惯用法。

\subsubsubsection{6.2.4\hspace{0.2cm}使用复制和交换习惯用法提供异常安全}

正如上一节提到的,因为\texttt{Array}类定义了析构函数,根据5规则,它还应该定义其他特殊的成员函数。本节中,将了解一个习语,该习语可以在没有样板文件的情况下完成此任务,同时还添加了强大的异常安全性。

如果不熟悉异常安全级别,这里有一个函数和类型异常安全级别的快速回顾:

\begin{itemize}
\item 
不保证:这是最基本的水平。在使用对象时抛出异常后,不能保证对象的状态。

\item 
基本异常安全:可能会有副作用,但对象不会泄漏任何资源,将处于有效状态,并将包含有效数据(不一定与操作之前相同)。自定义类型至少应该提供这个级别的异常安全。

\item 
强异常安全性:无副作用。对象的状态将与操作之前相同。

\item 
无抛出保证:操作总是成功的。如果在操作期间抛出异常,将在内部捕获并处理它,这样操作就不会在外部抛出异常。这些操作可以标记为\texttt{noexcept}。
\end{itemize}

那么,怎样才能一石二鸟,写出无模板的特殊成员,同时又提供强大的异常安全性呢?其实很简单。当有\texttt{swap}函数后,就可以用它来实现赋值操作符:

\begin{lstlisting}[style=styleCXX]
Array &operator=(Array other) noexcept {
	swap(*this, other);
	return *this;
}
\end{lstlisting}

在例子中,一个操作符就可以同时完成复制和移动赋值。在复制的情况下,通过值来获取参数,因此这是进行临时复制的地方。然后,要做的就是交换成员。这里不仅实现了强异常安全性,而且还能够不抛出赋值操作符的主体。但是,当复制发生时,仍然可以在调用函数之前抛出异常。在移动赋值的情况下,没有复制,因为按值获取将只获取移动的对象。

现在,定义复制构造函数:

\begin{lstlisting}[style=styleCXX]
Array(const Array &other) : array_{new T[other.size_]},
size_{other.size_} {
	std::copy_n(other.array_, size_, array_);
}
\end{lstlisting}

这个可以根据\texttt{T}来抛出,因为它分配了内存。再来定义移动构造函数:

\begin{lstlisting}[style=styleCXX]
Array(Array &&other) noexcept
	: array_{std::exchange(other.array_, nullptr)},
size_{std::exchange(other.size_, 0)} {}
\end{lstlisting}

这里,使用\texttt{std::exchange}以便初始化成员,清除其他成员,所有这些都在初始化列表中。构造函数没有声明,除非出于性能原因。例如,只有当元素不可移动构造时,\texttt{std::vector}才能在增长时移动,否则将复制。

就是这样!这里已经创建了一个数组类,它提供了强大的异常安全性,而且只需要很少的工作,也没有代码重复。

现在来处理另一个C++习惯用法,可以在标准库中的一些地方找到它。

\subsubsubsection{6.2.5\hspace{0.2cm}编写niebloid}

Niebloids以Eric Niebler的名字命名,是一种函数对象类型,从C++17开始,标准将其用于定制点。随着第5章中描述的标准范围的引入,它们开始流行起来,但它们是Niebler在2014年首次提出的。目的是在不需要ADL的地方禁用ADL,这样编译器就不会考虑来自其他名称空间的重载。还记得前几节中的两步习语吗?因为不方便且容易忘记,所以引入了自定义点对象的概念。本质上,这些是执行两步操作的函数对象。

如果库需要提供定制点,那么使用niebloids实现可能是个好主意。C++17及以后引入的标准库中的所有定制点都以这种方式实现是有原因的。即使只需要创建一个函数对象,也可以考虑使用niebloids。它们提供了ADL的所有优点,同时减少了缺点。它们允许特化,并且与概念一起提供了一种自定义可调用对象的重载集的方法。还允许更好地定制算法,这一切的代价是编写比更冗长的代码。

本节中,将创建一个简单的范围算法,并将其实现为niebloid。我们称它为contains,因为它只返回一个布尔值,表示给定的元素是否在范围内。首先,让创建函数对象本身,首先声明其基于迭代器的调用操作符:

\begin{lstlisting}[style=styleCXX]
namespace detail {
	struct contains_fn final {
		template <std::input_iterator It, std::sentinel_for<It> Sent, typename T,
			typename Proj = std::identity>
		requires std::indirect_binary_predicate<
			std::ranges::equal_to, std::projected<It, Proj>, const T *> constexpr
		bool
			operator()(It first, Sent last, const T &value, Proj projection = {})
		const {
\end{lstlisting}

它看起来很冗长,但所有这些代码都是有目的的。使用\texttt{final}结构来帮助编译器生成更高效的代码。如果查看模板形参,会看到一个迭代器和一个哨兵——每个标准范围的基本构建块。哨兵通常是一个迭代器,可以与迭代器进行比较的半正则类型(半正则类型是可复制和可默认初始化的)。接下来,\texttt{T}是要搜索的元素类型,而\texttt{Proj}表示投影——在比较之前应用于每个范围元素的操作(\texttt{std::identity}的默认值只是将其输入作为输出传递)。

在模板参数之后,还有对它们的要求,运算符可以比较投影值和搜索值是否相等。在这些约束之后,只需指定函数参数即可。

来看看它是如何实现的:

\begin{lstlisting}[style=styleCXX]
	while (first != last && std::invoke(projection, *first) != value)
		++first;
	return first != last;
}
\end{lstlisting}

即使没有使用标准范围,函数也可以工作;还需要一个重载的范围。其声明可如下:

\begin{lstlisting}[style=styleCXX]
template <std::ranges::input_range Range, typename T,
	typename Proj = std::identity>
requires std::indirect_binary_predicate<
	std::ranges::equal_to,
	std::projected<std::ranges::iterator_t<Range>, Proj>,
	const T *> constexpr bool
operator()(Range &&range, const T &value, Proj projection = {}) const {
\end{lstlisting}

这一次,使用满足\texttt{input\_range}的概念、元素值和投影类型的类型作为模板参数。要求调用投影后的范围迭代器可以与\texttt{T}类型的对象比较是否相等,类似于前面的操作。最后,使用范围、值和投影作为重载的参数。

这个操作符的主体也非常简单:

\begin{lstlisting}[style=styleCXX]
		return (*this)(std::ranges::begin(range), std::ranges::end(range),
		value,
			std::move(projection));
		}
	};
} // namespace detail
\end{lstlisting}

只需使用给定范围内的迭代器和哨兵调用前面的重载,同时传递值和未更改的投影。现在,对于最后一部分,不仅仅是\texttt{contains\_fn}可调用对象,还需要提供一个包含niebloid:

\begin{lstlisting}[style=styleCXX]
inline constexpr detail::contains_fn contains{};
\end{lstlisting}

通过声明类型为\texttt{contains}的内联变量\texttt{contains\_fn},允许其他人使用变量名调用niebloid。现在,来调用它,看看它是否有效:

\begin{lstlisting}[style=styleCXX]
int main() {
	auto ints = std::ranges::views::iota(0) | std::ranges::views::take(5);
	
	return contains(ints, 42);
}
\end{lstlisting}

就是这样!ADL约束函子会按预期的方式工作。

如果认为所有这些都有点过于冗长,那么可能会对\texttt{tag\_invoke}感兴趣,它可能在将来的某个时候成为标准的一部分。请参阅进一步阅读部分,以获得关于这个主题的论文和一个YouTube视频,它很好地解释了ADL、niebloids、隐藏友元和\texttt{tag\_invoke}。

现在,来了解一下另一个C++的习惯用法。

\subsubsubsection{6.2.6\hspace{0.2cm}基于策略的设计风格}

基于策略的设计是由Andrei Alexandrescu在他的著作《Modern C++ design》中首次提出的。虽然这本书出版于2001年,但其中的许多观点至今仍在使用。可以在本章末尾的扩展阅读部分找到它的链接。策略习语基本上是一个编译时等效“四人帮”的策略模式。如果需要编写具有可定制行为的类,可以将其作为模板参数,并将适当的策略作为模板参数。一个实际的例子可能是标准的分配器,它作为最后一个模板参数作为策略传递给许多C++容器。

返回Array类,并添加一个用于调试打印的策略:

\begin{lstlisting}[style=styleCXX]
template <typename T, typename DebugPrintingPolicy = NullPrintingPolicy>
class Array {
\end{lstlisting}

可以使用不打印任何内容的默认策略,\texttt{NullPrintingPolicy}可以进行如下实现:

\begin{lstlisting}[style=styleCXX]
struct NullPrintingPolicy {
	template <typename... Args> void operator()(Args...) {}
};
\end{lstlisting}

不管给出什么参数,它都不会做任何事情。编译器将完全优化它,所以当调试打印功能没有使用时,不会支付任何开销。

如果希望类更详细一些,可以使用不同的策略:

\begin{lstlisting}[style=styleCXX]
struct CoutPrintingPolicy {
	void operator()(std::string_view text) { std::cout << text << std::endl;
}
};
\end{lstlisting}

这次,将简单地将传递给策略的文本打印到\texttt{cout}。还需要修改类来使用策略:

\begin{lstlisting}[style=styleCXX]
Array(T *array, int size) : array_{array}, size_{size} {
	DebugPrintingPolicy{}("constructor");
}

Array(const Array &other) : array_{new T[other.size_]},
size_{other.size_} {
	DebugPrintingPolicy{}("copy constructor");
	std::copy_n(other.array_, size_, array_);
}
// ... other members ...
\end{lstlisting}

只需调用策略的\texttt{operator()},传递要打印的文本。因为策略是无状态的,所以可以在每次需要使用时实例化,而不需要额外的成本。另一种方法是调用它的静态函数。

现在,需要做的就是实例化\texttt{Array}类,使用所需的策略:

\begin{lstlisting}[style=styleCXX]
Array<T, CoutPrintingPolicy>(new T[size], size);
\end{lstlisting}

使用编译时间策略的缺点是,使用不同策略的模板实例化具有不同的类型。这意味着需要更多的工作,例如:从常规的\texttt{Array}类分配给一个具有\texttt{CoutPrintingPolicy}的类。为此,需要将赋值操作符实现为模板函数,并将策略作为模板参数。

有时,使用策略的另一种选择是使用特征。例如,以\texttt{std::iterator\_traits}为例,在编写使用迭代器的算法时,可以使用它们来使用关于迭代器的各种信息。一个例子可以是\texttt{std::iterator\_traits<T>::value\_type},既可以用于定义\texttt{value\_type}成员的自定义迭代器,也可以用于简单的指针(这种情况下,\texttt{value\_type}将指向指向的类型)。

关于基于策略的设计的内容已经够多了。列表中的下一个,是可以应用于多种场景的强大习惯用法。








