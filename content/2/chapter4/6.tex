

尽管部署服务听起来很简单,如果仔细观察,会发现还有很多事情需要考虑。本节将描述如何执行有效的部署、在安装服务后配置服务、在部署后检查服务是否保持正常运行,以及如何在最小化宕机时间内的同时完成这些工作。

\subsubsubsection{4.6.1\hspace{0.2cm}边车设计模式}

还记得本章前面提到的Envoy吗?对于高效的应用程序开发来说,它是一个非常有用的工具。可以将Envoy代理与应用程序一起部署,就像在摩托车旁边部署挎车一样,而不是将日志、监控或网络等基础设施服务嵌入到应用程序中,加在一起会比没有sidekick(这种模式的另一种叫法)的应用做得更多。

使用边车可以加速开发,因为它带来的许多功能需要由微服务独立开发。因为边车独立于应用程序,所以可以使用适合的编程语言进行开发。边车及其提供的所有功能,都可以由独立的开发团队进行维护,并独立于主服务进行更新。

因为边车就在需要增强的应用程序旁边,它们可以使用本地的进程进行通信。通常,与与另一个主机通信相比,这就已经足够快,而且要快得多。但请记住,这有时可能是一个巨大的负担。

即使部署了第三方服务,将所选的边车部署到旁边仍然有用:可以监视资源使用情况,以及主机和服务的状况,还可以在整个分布式系统中跟踪请求。有时也可以根据服务的情况,通过编辑配置文件或Web界面动态地配置服务。

\hspace*{\fill} \\ %插入空行
\noindent
\textbf{使用Envoy部署带有跟踪和反向代理的服务}

现在让Envoy作为我们部署的前线代理人。首先创建Envoy的配置文件,在例子中名为\textit{envoy-front\_proxy.yaml},代理地址为:

\begin{tcblisting}{commandshell={}}
static_resources:
  listeners:
    - address:
      socket_address:
        address: 0.0.0.0
        port_value: 8080
      traffic_direction: INBOUND
\end{tcblisting}

这里已经指定Envoy将侦听端口8080上的传入信息。在配置后,将把信息转到我们的服务中。现在,让指定使用我们的服务实例集来处理HTTP请求,并在上面添加一些跟踪功能。首先,让添加一个HTTP端点:

\begin{tcblisting}{commandshell={}}
filter_chains:
  - filters:
  - name: envoy.filters.network.http_connection_manager
    typed_config:
      "@type":
type.googleapis.com/envoy.extensions.filters.network.
http_connection_manager.v3.HttpConnectionManager
\end{tcblisting}

现在,指定请求分配ID,并由分布式跟踪系统Jaeger跟踪:

\begin{tcblisting}{commandshell={}}
generate_request_id: true
tracing:
  provider:
    name: envoy.tracers.dynamic_ot
    typed_config:
      "@type":
type.googleapis.com/envoy.config.trace.v3.DynamicOtConfig
     library: /usr/local/lib/libjaegertracing_plugin.so
     config:
       service_name: front_proxy
       sampler:
         type: const
         param: 1
       reporter:
         localAgentHostPort: jaeger:6831
       headers:
         jaegerDebugHeader: jaeger-debug-id
         jaegerBaggageHeader: jaeger-baggage
         traceBaggageHeaderPrefix: uberctx-
       baggage_restrictions:
         denyBaggageOnInitializationFailure: false
         hostPort: ""
\end{tcblisting}

将为请求创建ID,并在本地Jaeger插件中使用OpenTracing标准(\textit{DynamicOtConfig})。该插件将生成一个Jaeger实例在指定地址下运行,并添加指定的头文件。

还需要指定来自所有域名的所有信息(参见匹配部分)都应转到我们的服务集群:

\begin{tcblisting}{commandshell={}}
codec_type: auto
stat_prefix: ingress_http
route_config:
  name: example_route
  virtual_hosts:
    - name: front_proxy
      domains:
        - "*"
\end{tcblisting}
\begin{tcblisting}{commandshell={}}
      routes:
        - match:
            prefix: "/"
          route:
            cluster: example_service
          decorator:
            operation: example_operation
\end{tcblisting}

这里定义了\textit{example\_service}集群。注意,到达集群的每个请求都将由预定义的操作修饰符标记。还需要指定路由器地址:

\begin{tcblisting}{commandshell={}}
http_filters:
- name: envoy.filters.http.router
  typed_config: {}
use_remote_address: true
\end{tcblisting}

现在知道了如何处理和跟踪请求,剩下的就是定义使用的集群。先从我们的服务集群开始:

\begin{tcblisting}{commandshell={}}
clusters:
  - name: example_service
    connect_timeout: 0.250s
    type: strict_dns
    lb_policy: round_robin
    load_assignment:
      cluster_name: example_service
      endpoints:
        - lb_endpoints:
        - endpoint:
            address:
              socket_address:
                address: example_service
                port_value: 5678
\end{tcblisting}

每个集群可以有多个实例(端点)。这里,若决定添加更多端点,则将使用循环策略对传入请求进行负载平衡。

再添加一个管理界面:

\begin{tcblisting}{commandshell={}}
admin:
  access_log_path: /tmp/admin_access.log
  address:
   socket_address:
     address: 0.0.0.0
     port_value: 9901
\end{tcblisting}

现在将配置放在一个容器中,它将使用Dockerfile运行Envoy,将其命名为\textit{Dockerfile-front\_proxy}:

\begin{tcblisting}{commandshell={}}
FROM envoyproxy/envoy:v1.17-latest

RUN apt-get update && \
apt-get install -y curl && \
rm -rf /var/lib/apt/lists/*
RUN curl -Lo -
https://github.com/tetratelabs/getenvoy-package/files/3518103/getenvoy-cent
os-jaegertracing-plugin.tar.gz | tar -xz && mv libjaegertracing.so.0.4.2
/usr/local/lib/libjaegertracing_plugin.so

COPY envoy-front_proxy.yaml /etc/envoy/envoy.yaml

\end{tcblisting}

还下载了在Envoy配置中使用的Jaeger本地插件。

现在指定如何使用Docker Compose在几个容器中运行代码。从前端代理服务定义开始,创建一个\textit {docker-compose.yaml}文件:

\begin{tcblisting}{commandshell={}}
version: "3.7"

services:
  front_proxy:
    build:
      context: .
      dockerfile: Dockerfile-front_proxy
\end{tcblisting}
\begin{tcblisting}{commandshell={}}
    networks:
      - example_network
    ports:
      - 12345:12345
      - 9901:9901
\end{tcblisting}

这里使用Dockerfile,这是一个简单的网络,在主机上公开了容器的两个端口:服务和管理接口。现在让添加代理要指向的服务:

\begin{tcblisting}{commandshell={}}
example_service:
  image: hashicorp/http-echo
  networks:
    - example_network
  command: -text "It works!"

\end{tcblisting}

在例子中,服务只会在一个简单的Web服务器中显示一个预定义的字符串。

现在,在另一个容器中运行Jaeger,将其端口暴露给外部:

\begin{tcblisting}{commandshell={}}
jaeger:
  image: jaegertracing/all-in-one
  environment:
    - COLLECTOR_ZIPKIN_HTTP_PORT=9411
  networks:
    - example_network
  ports:
    - 16686:16686

\end{tcblisting}

最后一步是定义网络:

\begin{tcblisting}{commandshell={}}
  networks:
    example_network: {}
\end{tcblisting}

完成了。现在可以使用\textit{docker-compose up -\,-build}运行这个服务,并将浏览器指向特定的端口。

使用边车代理还有一个好处:即使服务死机,边车通常是活着的,可以在主服务关闭时响应外部请求。当服务重新部署(例如,由于更新)时,这同样适用。说到这里,让我们了解如何最小化相关的宕机时间。

\subsubsubsection{4.5.2\hspace{0.2cm}零宕机部署}

有两种常见的方法来最小化部署期间的宕机风险:\textbf{蓝绿部署}和\textbf{金丝雀发布}。两者都可以使用Envoy边车模式。

\hspace*{\fill} \\ %插入空行
\noindent
\textbf{蓝绿部署}

\textbf{蓝绿部署}可以最小化宕机时间和与部署应用程序的相关风险。为此,需要两个相同的生产环境,称为蓝色和绿色。绿色服务于客户,可以执行蓝色的更新。一旦进行了更新,服务进行了测试,所有看起来都很稳定,可以切换流量,让其流到更新的(蓝色)环境。

如果在切换后蓝色环境中发现了问题,绿色环境仍然存在——可以将切换回来。用户甚至可能不会注意到,而且由于两个环境都已启动并运行,在切换期间应该不会停机。只要确保在切换期间不会丢失任何数据(例如,在新环境中进行的事务)即可。

\hspace*{\fill} \\ %插入空行
\noindent
\textbf{金丝雀发布}

要避免所有服务实例在更新后都失败,最简单的方法是,不要一次更新所有服务实例。这就是蓝绿部署的增量变体背后的关键思想,也称为\textbf{金丝雀发步}。

在Envoy中,可以在配置的\textit{routes}部分中输入以下内容:

\begin{tcblisting}{commandshell={}}
- match:
    prefix: "/"
  route:
    weighted_clusters:
      clusters:
      - name: new_version
        weight: 5
      - name: old_version
        weight: 95
\end{tcblisting}

还应该记得在前面的代码片段中定义两个集群,第一个是旧版服务:

\begin{tcblisting}{commandshell={}}
clusters:
  - name: old_version
    connect_timeout: 0.250s
    type: strict_dns
    lb_policy: round_robin
\end{tcblisting}
\begin{tcblisting}{commandshell={}}
    load_assignment:
      cluster_name: old_version
      endpoints:
        - lb_endpoints:
          - endpoint:
            address:
              socket_address:
                address: old_version
                port_value: 5678

\end{tcblisting}

第二个集群将运行新版本:

\begin{tcblisting}{commandshell={}}
- name: new_version
  connect_timeout: 0.250s
  type: strict_dns
  lb_policy: round_robin
  load_assignment:
    cluster_name: new_version
    endpoints:
      - lb_endpoints:
        - endpoint:
            address:
              socket_address:
                address: new_version
                port_value: 5678
\end{tcblisting}

当一个部署更新时,服务的新版本将只会让一小部分(这里:5\%)的用户看到和使用。如果更新的实例保持稳定,并且没有检查和验证失败,那么可以分几个步骤逐步更新越来越多的主机,直到所有主机都切换到新版本。可以通过手动更新配置文件或使用管理端点来实现。

现在,来了解一下最后一个部署模式。

\subsubsubsection{4.5.3\hspace{0.2cm}外部配置存储}

如果部署的是一个简单的应用程序,可以只部署其配置。然而,当希望有一个包含许多应用程序实例的更复杂的部署时,为了重新配置而重新部署新版本的应用程序很快就会成为一种负担,从而手动配置更改是不可取的。引入外部配置存储是解决问题的好方法。

从本质上说,应用程序可以从上述商店获取配置,而不是仅仅依赖于它们的本地配置文件。这可以为多个实例提供公共设置,并为其中一些实例调优参数,同时拥有一种简单而集中的方式来监视所有配置。如果想让仲裁程序决定哪些节点将作为主节点,哪些节点将作为备份节点,外部配置存储可以为实例提供此类信息。实现配置更新过程也很有用,这样可以在操作期间轻松地重新配置实例。可以使用现成的解决方案,如Firebase Remote Config,利用基于Java的Netflix Archaius,或者编写一个配置存储,利用云存储和更改通知。

已经了解了一些有用的部署模式,接下来转向另一个重要的高级设计主题:API。



























