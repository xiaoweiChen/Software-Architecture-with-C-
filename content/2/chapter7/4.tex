
有几种方法可以获取所依赖的外部项目。例如,可以将它们添加为一个Conan依赖,使用CMake的\texttt{find\_package}来查找操作系统提供的或以其他方式安装的版本,或者自己获取和编译依赖。

这一节的关键是:如果可以,使用Conan。这样,就可以使用与项目及其依赖项需求相匹配的依赖项版本。

如果目标是支持多个平台,甚至同一发行版本的多个版本,使用Conan或自行编译都是可行的方法。这样,无论您在哪个操作系统上编译,都可以使用相同的依赖版本。

现在来讨论一下获取CMake本身提供的依赖项的几种方法,然后跳转到使用Conan的多平台包管理器。

\subsubsubsection{7.4.1\hspace{0.2cm}获取依赖项}

从源代码中准备依赖项的一种方法是使用CMake的内置\texttt{FetchContent}模块。它可以下载依赖项,然后作为常规目标为对其进行构建。

新特性出现在CMake 3.11中。它是\texttt{ExternalProject}模块的替代品,\texttt{ExternalProject}模块有很多缺陷。其中之一是,构建期间克隆外部存储库,因此CMake无法推断外部项目定义的目标,及其依赖关系。这使得许多项目需要手动定义此类外部目标的\texttt{include}目录和\texttt{lib}路径,从而完全忽略所需接口编译标志和依赖关系。而\texttt{FetchContent}就没有这样的问题,所以推荐使用。

\begin{tcolorbox}[colback=blue!5!white,colframe=blue!75!black, title=Note]
\hspace*{0.7cm}展示如何使用之前,必须知道\texttt{FetchContent}和\texttt{ExternalProject}(以及使用Git子模块和类似的方法)都有一个重要的缺陷。如果有许多依赖项使用相同的第三方库本身,那么可能最终会拥有同一个项目的多个版本,例如Boost的几个版本。使用类似Conan的包管理器则可以避免此类问题的发生。
\end{tcolorbox}

作为示例,演示如何使用\texttt{FetchContent}特性将GTest集成到项目中。首先,创建一个\texttt{FetchGTest.cmake}文件,并把它放在源码树中的cmake目录中。\texttt{FetchGTest}脚本将有如下定义:

\begin{lstlisting}[style=styleCMake]
include(FetchContent)

FetchContent_Declare(
	googletest
	GIT_REPOSITORY https://github.com/google/googletest.git
	GIT_TAG dcc92d0ab6c4ce022162a23566d44f673251eee4)

FetchContent_GetProperties(googletest)
if(NOT googletest_POPULATED)
	FetchContent_Populate(googletest)
	add_subdirectory(${googletest_SOURCE_DIR} ${googletest_BINARY_DIR}
		EXCLUDE_FROM_ALL)
endif()

message(STATUS "GTest binaries are present at ${googletest_BINARY_DIR}")
\end{lstlisting}

首先,包含内置的\texttt{FetchContent}模块。加载模块后,使用\texttt{FetchContent\_Declare}声明该依赖,命名依赖项,并指定CMake克隆的存储库,以及相应的提交版本。

现在,读取外部库的属性并填充(如果还没有完成的话)。当有了源码,就可以使用\texttt{add\_subdirectory}进行处理。在运行make all等命令时,如果其他目标不需要这些目标,那么\texttt{EXCLUDE\_FROM\_ALL}选项会告诉CMake不要构建这些目标。在成功处理目录之后,脚本将打印一条消息,指出构建后GTests库将在其中放置的目录。

若不喜欢将依赖关系与项目一起构建,也许下一种集成依赖关系的方法将更适合。

\subsubsubsection{7.4.2\hspace{0.2cm}find脚本}

假设依赖项在本地上的某个地方可用,只需使用\texttt{find\_package}尝试搜索它即可。如果依赖项提供了配置或目标文件(稍后会详细介绍),那么只需要编写这个简单的命令就可以了。当然,前提是依赖项在本地可用。如果不可用,使用者有责任在项目运行CMake之前安装它们。

要创建上述文件,依赖项将需要使用CMake,但并非总是如此。如何处理那些不使用CMake的库?如果该库很受欢迎,那么很可能已经有人创建了find脚本供他人使用。1.70以上版本的Boost库就是这种方法的一个常见示例。CMake附带了一个FindBoost模块,可以通过\texttt{find\_package(Boost)}来执行。

要使用前面的模块找到Boost,首先需要在系统上安装它。之后,在CMake列表中,应该设置合理的选项。例如使用动态和多线程Boost库,而不是静态链接到C++运行时,参考以下代码:

\begin{lstlisting}[style=styleCMake]
set(Boost_USE_STATIC_LIBS OFF)
set(Boost_USE_MULTITHREADED ON)
set(Boost_USE_STATIC_RUNTIME OFF)
\end{lstlisting}

然后,搜索库,如下所示:

\begin{lstlisting}[style=styleCMake]
find_package(Boost 1.69 EXACT REQUIRED COMPONENTS Beast)
\end{lstlisting}

这里只使用Beast,它是Boost的一部分,是一个很棒的网络库。当找到该库,可以链接到目标,代码如下所示:

\begin{lstlisting}[style=styleCMake]
target_link_libraries(MyTarget PUBLIC Boost::Beast)
\end{lstlisting}

现在已经知道了如何正确地使用find脚本,接下来了解如何编写find脚本。

\subsubsubsection{7.4.3\hspace{0.2cm}编写find脚本}

如果依赖项既没有提供配置文件和目标文件,也没有人为它编写find模块,那可以自己编写这样的模块。

这应该不会是经常要做的事情,所以本小节如果不需要可以略过。要获得更多的描述,也应该阅读CMake官方文档中的指南(链接在扩展阅读部分)或只看一些安装了CMake的模块(通常在Unix系统上的目录如\texttt{/usr/share/cmake-3.17/Modules})。简单起见,假设只需要找到一个依赖项的配置,但是可以分别找到Release和Debug二进制文件。这将使设置不同的目标,关联不同的变量。

脚本名决定了你将传递给\texttt{find\_package},如果希望以\texttt{find\_package(Foo)}结束,脚本应该命名为\texttt{FindFoo.cmake}。

一个好的方式是用reStructuredText部分开始脚本,描述脚本实际将做什么,将设置哪些变量等。这种描述的个例子如下所示:

\begin{tcblisting}{commandshell={}}
#.rst:
# FindMyDep
# ----------
#
# Find my favourite external dependency (MyDep).
#
# Imported targets
# ^^^^^^^^^^^^^^^^
#
# This module defines the following :prop_tgt:`IMPORTED` target:
#
# ``MyDep::MyDep``
# The MyDep library, if found.
#
\end{tcblisting}

通常,还需要描述脚本将设置的变量:

\begin{tcblisting}{commandshell={}}
# Result variables
# ^^^^^^^^^^^^^^^^
#
# This module will set the following variables in your project:
#
# ``MyDep_FOUND``
# whether MyDep was found or not
# ``MyDep_VERSION_STRING``
# the found version of MyDep
\end{tcblisting}

如果MyDep本身有依赖项,是时候找到它们了:

\begin{lstlisting}[style=styleCMake]
find_package(Boost REQUIRED)
\end{lstlisting}

现在可以开始搜索库了。常见的方法是使用\texttt{pkg-config}:

\begin{lstlisting}[style=styleCMake]
find_package(PkgConfig)
pkg_check_modules(PC_MyDep QUIET MyDep)
\end{lstlisting}

若\texttt{pkg-config}有关于依赖的信息,就会设置一些变量,可以用这种方式来找到依赖。一个好主意可能是让用户设置一个变量来指向库的位置。按照CMake约定,应该命名为\texttt{MyDep\_ROOT\_DIR}。

为CMake提供这个变量,用户可以使用\texttt{-DMyDep\_ROOT\_DIR=some/path}调用CMake,在他们的构建目录中修改CMakeCache.txt中的变量,或使用ccmake或cmake-gui。

现在,可以使用前面提到的路径来搜索依赖的头文件和库文件:

\begin{lstlisting}[style=styleCMake]
find_path(MyDep_INCLUDE_DIR
	NAMES MyDep.h
	PATHS "${MyDep_ROOT_DIR}/include" "${PC_MyDep_INCLUDE_DIRS}"
	PATH_SUFFIXES MyDep
)

find_library(MyDep_LIBRARY
	NAMES mydep
	PATHS "${MyDep_ROOT_DIR}/lib" "${PC_MyDep_LIBRARY_DIRS}"
)
\end{lstlisting}

然后,还需要设置找到的版本,就像在脚本头部中承诺的那样。要使用从pkg-config中找到的,可以这样写:

\begin{lstlisting}[style=styleCMake]
set(MyDep_VERSION ${PC_MyDep_VERSION})
\end{lstlisting}

或者,可以从头文件的内容、库路径的组件或使用其他方法提取版本。当这些做完后,可以使用CMake的内置脚本来决定是否成功找到库,同时处理\texttt{find\_package}所有可能的参数:

\begin{lstlisting}[style=styleCMake]
include(FindPackageHandleStandardArgs)
find_package_handle_standard_args(MyDep
	FOUND_VAR MyDep_FOUND
	REQUIRED_VARS
	MyDep_LIBRARY
	MyDep_INCLUDE_DIR
	VERSION_VAR MyDep_VERSION
	)
\end{lstlisting}

当决定提供一个目标而不仅仅是一堆变量时,是时候定义了:

\begin{lstlisting}[style=styleCMake]
if(MyDep_FOUND AND NOT TARGET MyDep::MyDep)
	add_library(MyDep::MyDep UNKNOWN IMPORTED)
	set_target_properties(MyDep::MyDep PROPERTIES
		IMPORTED_LOCATION "${MyDep_LIBRARY}"
		INTERFACE_COMPILE_OPTIONS "${PC_MyDep_CFLAGS_OTHER}"
		INTERFACE_INCLUDE_DIRECTORIES "${MyDep_INCLUDE_DIR}"
		INTERFACE_LINK_LIBRARIES Boost::boost
		)
endif()
\end{lstlisting}

最后,将内部使用的变量进行隐藏,不将其暴露给不想处理它们的用户:

\begin{lstlisting}[style=styleCMake]
mark_as_advanced(
	MyDep_INCLUDE_DIR
	MyDep_LIBRARY
)
\end{lstlisting}

现在,有了一个完整的find模块,可以按以下方式使用:

\begin{lstlisting}[style=styleCMake]
find_package(MyDep REQUIRED)
target_link_libraries(MyTarget PRIVATE MyDep::MyDep)
\end{lstlisting}

这就是自己编写find模块的方式。

\begin{tcolorbox}[colback=blue!5!white,colframe=blue!75!black, title=Note]
\hspace*{0.7cm}不要为拥有的包创建Find$ \ast $.cmake模块。这些是为不支持CMake的包准备的。不过,可以创建一个Config$ \ast $.cmake模块(本章后面内容)。
\end{tcolorbox}

现在,来展示如何使用合适的包管理器,而不是手动进行下载依赖。

\subsubsubsection{7.4.4\hspace{0.2cm}使用Conan包管理器}

Conan是一个开源的、分散的原生包管理器。它支持多种平台和编译器,可以与多个构建系统集成。

如果有包还没有为环境所构建,Conan将在机器上构建它,而不是下载已经构建的版本。构建完成时,可以将它上传到公共存储库、私有的\texttt{conan\_server}实例或Artifactory服务器。

\hspace*{\fill} \\ %插入空行
\noindent
\textbf{准备使用Conan}

如果第一次运行Conan,它将根据环境创建一个默认配置文件。可以通过创建一个新的配置文件,或更新默认配置文件来修改它的一些设置。假设正在使用Linux,并且希望使用GCC 9.x编译所有东西。则可以使用如下代码:

\begin{tcblisting}{commandshell={}}
conan profile new hosacpp
conan profile update settings.compiler=gcc hosacpp
conan profile update settings.compiler.libcxx=libstdc++11 hosacpp
conan profile update settings.compiler.version=10 hosacpp
conan profile update settings.arch=x86_64 hosacpp
conan profile update settings.os=Linux hosacpp
\end{tcblisting}

如果依赖来自于其他存储库而不是默认存储库,可以使用\texttt{conan remote add <repo> <repo\_url>}添加它们。例如,可以使用它来配置公司的一个库。

现在已经设置了Conan,接下来展示如何使用Conan抓取依赖,并将所有依赖集成到CMake脚本中。

\hspace*{\fill} \\ %插入空行
\noindent
\textbf{指定Conan的依赖}

项目依赖于C++ REST SDK。为了告诉Conan这一点,需要创建一个名为conanfile.txt的文件。在例子中,它将包含以下内容:

\begin{tcblisting}{commandshell={}}
[requires]
cpprestsdk/2.10.18

[generators]
CMakeDeps
\end{tcblisting}

可以在这里指定任意数量的依赖项。它们中的每一个都可以有一个固定版本、固定版本的范围或一个标签,比如latest。在@符号之后,可以找到拥有包的公司和可以选择包的特定版本(通常是稳定和测试)的通道。

生成器部分是指定想要使用的构建系统的地方。对于CMake项目,应该使用CMakeDeps。还可以生成许多其他的工具,包括用于生成编译器参数、CMake工具链文件、Python虚拟环境等的工具。

在例子中,没有指定其他的选项,但可以很容易地进行添加,并为包及其依赖项配置变量。例如,要将依赖项编译为静态库,可以这样写:

\begin{tcblisting}{commandshell={}}
[options]
cpprestsdk:shared=False
\end{tcblisting}

完成了conanfile.txt后,就可以让Conan使用它了。

\hspace*{\fill} \\ %插入空行
\noindent
\textbf{安装Conan的依赖}

要在CMake代码中使用Conan包,必须先安装。Conan中,这意味着下载源代码并构建它们或下载预构建的二进制文件,以及创建将在CMake中使用的配置文件。要让Conan在创建构建目录后处理这个,应该跳转到其目录下,并运行以下命令:

\begin{tcblisting}{commandshell={}}
conan install path/to/directory/containing/conanfile.txt --build=missing -s
build_type=Release -pr=hosacpp
\end{tcblisting}

默认情况下,Conan希望将所有依赖项下载为预构建的二进制文件。如果服务器没有预先构建,Conan将进行构建,而不是在传递\texttt{-\,-build=missing}时退出。这会告诉Conan获取使用与概要文件中相同的编译器和环境构建的发布版本。通过简单地调用\texttt{build\_type}设置为其他CMake构建类型的另一个命令,这里可以为多个构建类型安装包。如果需要的话,这还可以在多个版本间快速的切换。如果使用默认配置文件(Conan可以自动检测的配置文件),就不要使用\texttt{-pr}。

若计划使用的CMake生成器没有在conanfile.txt中指定,可以将它添加到前面的命令中。例如,要使用\texttt{compiler\_args}生成器,应该添加\texttt{-\,-generator compiler\_args}。之后,可以使用它传递\texttt{@conanbuildinfo.args}生成的内容到编译器端。

\hspace*{\fill} \\ %插入空行
\noindent
\textbf{使用CMake的Conan目标}

当Conan完成下载、构建和配置依赖项,则需要告诉CMake对它们进行使用。

如果使用Conan与CMakeDeps生成器,一定要为\texttt{CMAKE\_BUILD\_TYPE}指定一个值。其他情况下,CMake将无法使用由Conan配置的包。调用(来自运行Conan的同一个目录)方法可以如下所示:

\begin{tcblisting}{commandshell={}}
cmake path/to/directory/containing/CMakeLists.txt -DCMAKE_BUILD_TYPE=Release
\end{tcblisting}

这样,就可以在发布模式中构建项目,从而必须使用Conan安装的一个类型。要找到依赖项,可以使用CMake的\texttt{find\_package}:

\begin{lstlisting}[style=styleCMake]
list(APPEND CMAKE_PREFIX_PATH "${CMAKE_BINARY_DIR}")
find_package(cpprestsdk CONFIG REQUIRED)
\end{lstlisting}

首先,将根构建目录添加到CMake找到包配置文件的路径。然后,找到Conan生成的包配置文件。

要将Conan定义的目标作为目标的依赖项,最好使用有命名空间的目标名:

\begin{lstlisting}[style=styleCMake]
target_link_libraries(libcustomer PUBLIC cpprestsdk::cpprest)
\end{lstlisting}

在CMake的配置过程中没有找到包时,会得到一个错误。如果没有别名,将在尝试链接时得到一个错误。

现在已经按照想要的方式编译并链接了目标,是时候对进行测试了。

\subsubsubsection{7.4.5\hspace{0.2cm}添加测试}

CMake有自己的测试驱动程序CTest。从CMakeLists添加新的测试套件很容易,可以使用自己写的,也可以使用测试框架提供的。本书的后面部分,将深入讨论测试,这里先展示如何基于GoogleTest或GTest测试框架快速且干净地添加单元测试。

通常,要在CMake中已写好的测试,需要添加以下内容:

\begin{lstlisting}[style=styleCMake]
if(CMAKE_PROJECT_NAME STREQUAL PROJECT_NAME)
	include(CTest)
	if(BUILD_TESTING)
		add_subdirectory(test)
	endif()
endif()
\end{lstlisting}

前面的代码段将首先检查是否正在构建的主要项目。通常,只想为项目运行测试,甚至省略对使用的第三方组件构建测试。这就是检查项目名称的原因。

如果要运行测试,则需要包含CTest模块。这将加载CTest提供的整个测试基础设施,定义它的附加目标,并调用一个名为\texttt{enable\_testing}的CMake函数,它将在其他事情中启用\texttt{BUILD\_TESTING}标志。这个标志可以缓存,所以可以在构建项目时禁用所有的测试,只要在生成构建系统时传递\texttt{-DBUILD\_TESTING=OFF}即可。

所有这些缓存变量都存储在构建目录中名为CMakeCache.txt的文本文件中。随意修改变量,改变CMake所做的。其不会覆盖文本中的设置,除非删除文件。可以使用ccmake、cmake-gui或手工完成。

如果\texttt{BUILD\_TESTING}为true,只需处理测试目录中的CMakeLists.txt文件即可。可以是这样的:

\begin{lstlisting}[style=styleCMake]
include(FetchGTest)
include(GoogleTest)

add_subdirectory(customer)
\end{lstlisting}

第一个包含调用前面描述的GTest的脚本。获取GTest后,当前的CMakeLists.txt通过\texttt{include(GoogleTest)},加载GoogleTest模块中定义的一些帮助函数。这能够更容易地将测试集成到CTest中。最后,通过\texttt{add\_subdirectory(customer)}来告诉CMake包含一些测试目录。

\texttt{test/customer/CMakeLists.txt}文件将添加一个带有测试的可执行文件,该可执行文件是用预定义的标志集和测试模块和GTest的链接编译的。然后,调用CTest助手函数来发现定义的测试。所有这些只是四行CMake代码:

\begin{lstlisting}[style=styleCMake]
add_executable(unittests unit.cpp)
target_compile_options(unittests PRIVATE ${BASE_COMPILE_FLAGS})
target_link_libraries(unittests PRIVATE domifair::libcustomer gtest_main)
gtest_discover_tests(unittests)
\end{lstlisting}

瞧!

现在,可以通过进入构建目录并使用以下命令来构建和执行测试:

\begin{tcblisting}{commandshell={}}
cmake --build . --target unittests
ctest # or cmake --build . --target test
\end{tcblisting}

可以为CTest传递\texttt{-j}标志。其工作原理就像Make或Ninja一样——并行执行测试。如果希望使用更短的构建命令,只需运行构建系统,即直接调用make就好。

\begin{tcolorbox}[colback=blue!5!white,colframe=blue!75!black, title=Note]
\hspace*{0.7cm}脚本中,最好使用较长的命令形式。这将使脚本独立于所使用的构建系统。
\end{tcolorbox}

当测试通过后,就可以考虑将它们提供给更广泛的受众了。






