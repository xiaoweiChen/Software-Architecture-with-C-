\begin{flushright}
	\zihao{0} 前言
\end{flushright}

现代C++可以在不降低可读性和可维护性的情况下,编写高性能应用程序。软件架构不仅仅是语言,本书将展示如何设计和构建健壮、可扩展,且性能良好的应用程序。

我们将从理解体系结构的重要性开始,逐步了解基本概念、示例和具体问题后,再对实际应用的案例进行研究。

将了解在独立应用程序中使用已有的设计模式,并探索可使应用程序更健壮、更安全、高性能和可维护的方式,再使用面向服务的架构、微服务、容器和无服务器技术等模式,对多个独立应用程序进行连接的高级服务。

阅读完本书,将能使用现代C++和相关工具构建分布式服务,并以客户推荐的解决方案进行交付。

有兴趣成为一名软件架构师,或者了解更多关于架构的趋势吗?如果答案是“是”,那么这本书应该对你有所帮助!

\hspace*{\fill} \\ %插入空行
\noindent\textbf{适读人群}

为了帮助使用现代C++的开发者将能够将知识运用到软件架构的实践指南中,本书采取了实践的方法来实现相关方案,会让读者们感觉干货满满。

\hspace*{\fill} \\ %插入空行
\textbf{本书内容}

\textit{第1章,软件架构的重要性和设计原则}。首先,为什么要设计软件。

\textit{第2章,架构风格}。介绍了在架构方面可以使用的方法。

\textit{第3章,功能和非功能需求}。探索客户的需求。

\textit{第4章,架构与系统设计}。创建有效的软件解决方案。

\textit{第5章,C++特性}。

\textit{第6章,设计模式和C++}。重点关注现代C++的习惯用法和代码构造。

\textit{第7章,构建和打包}。将代码投入产线。

\textit{第8章,编写可测试的代码}。如何在客户之前找到错误。

\textit{第9章,持续集成和持续部署}。自动化软件发布。

\textit{第10章,代码和部署的安全性}。确保系统不被破坏。

\textit{第11章,性能}。了解性能,看C++有多快——还能更快?

\textit{第12章,面向服务的架构}。基于服务构建系统。

\textit{第13章,设计微服务}。

\textit{第14章,容器}。为构建、打包和运行应用程序提供了统一的接口。

\textit{第15章,原生云设计}。超越传统基础设施,探索原生云上的设计。

\hspace*{\fill} \\ %插入空行
\textbf{编译环境}

本书中的代码示例主要是为GCC 10编写的,使用Clang或Microsoft Visual C++编译器也没什么问题,不过在旧编译器中可能缺少C++20的某些特性。为了获得一个尽可能接近作者的开发环境,我们建议在一个类似Linux的环境中使用Nix (\url{https://nixos.org/download.html})和direnv (\url{https://direnv.net/})。如果在一个包含示例的目录中运行\texttt{direnv allow},这两个工具会为你配置编译器和安装相关的依赖包。

没有Nix和direnv的话,就不能保证示例将正确地工作。如果使用macOS,Nix应该可以正常工作。如果在Windows上,Windows子系统Linux 2中一个Linux开发环境,可以与Nix一起使用。

要安装这两个工具,可以运行以下命令:

\begin{tcblisting}{commandshell={}}
# Install Nix
curl -L https://nixos.org/nix/install | sh
# Configure Nix in the current shell
. $HOME/.nix-profile/etc/profile.d/nix.sh
# Install direnv
nix-env -i direnv
# Download the code examples
git clone
https://github.com/PacktPublishing/Hands-On-Software-Architecture-with-Cpp.
git
# Change directory to the one with examples
cd Hands-On-Software-Architecture-with-Cpp
# Allow direnv and Nix to manage your development environment
direnv allow
\end{tcblisting}

执行上述命令后,Nix应该下载并安装所有依赖项。这可能需要一些时间,但它有助于确保与作者使用的是完全相同的工具。

\textbf{如果正在使用这本书的数字版本,建议自行输入代码或通过GitHub库访问代码(链接在下一节中提供)。这样做将避免与复制和粘贴代码时出现的错误}

\hspace*{\fill} \\ %插入空行
\textbf{下载示例}

可以从GitHub网站\url{https://github.com/PacktPublishing/Software-Architecture-with-Cpp}下载本书的示例代码文件。如果代码有更新,会在现有的GitHub库中更新。

我们还有其他的代码包,还有丰富的书籍和视频目录,都在\url{https://github.com/PacktPublishing/}。去看看吧!

\hspace*{\fill} \\ %插入空行
\textbf{联系方式}

我们欢迎读者的反馈。

\textbf{反馈}:如果你对这本书的任何方面有疑问,需要在你的信息的主题中提到书名,并给发邮件到\url{customercare@packtpub.com}。

\textbf{勘误}:尽管我们谨慎地确保内容的准确性,但错误还是会发生。如果在本书中发现了错误,请向我们报告,将不胜感激。请访问\url{www.packtpub.com/support/errata},选择相应书籍,点击勘误表提交表单链接,并输入详细信息。

\textbf{盗版}:如果在互联网上发现任何形式的非法拷贝,非常感谢提供地址或网站名称。请通过\url{copyright@packt.com}与我们联系,并提供材料链接。

\textbf{如果对成为书籍作者感兴趣}:如果你对某主题有专长,又想写一本书或为之撰稿,请访问\url{authors.packtpub.com}。

\hspace*{\fill} \\ %插入空行
\textbf{欢迎评论}

请留下评论。当阅读书籍,为什么不在购买网站上留下评论呢?其他读者可以看到您的评论,并根据您的意见来做出购买决定。我们在Packt可以了解您对我们产品的看法,作者也可以看到您对他们撰写书籍的反馈。谢谢你!

想要了解Packt的更多信息,请访问\url{packt.com}。










