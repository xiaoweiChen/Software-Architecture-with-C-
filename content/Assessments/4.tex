\begin{enumerate}
\item
什么是事件源?
\begin{itemize}
\item 
这是一种架构模式,它依赖于跟踪改变系统状态的事件,而不是跟踪状态本身。它带来的好处包括较低的延迟、免费的审计日志和可调试性。
\end{itemize}

\item 
CAP定理的实际结果是什么?
\begin{itemize}
\item 
随着网络分区的发生,如果想要一个分布式系统,需要在一致性和可用性之间做出选择。对于分区,可以返回旧的数据、错误或冒超时的风险。
\end{itemize}

\item 
Netflix的Chaos Monkey能做什么?
\begin{itemize}
\item 
可以为服务意外停机做好准备。
\end{itemize}

\item 
缓存可以应用在哪里?
\begin{itemize}
\item 
在客户端,在Web服务器、数据库或应用程序的前面,或者在潜在客户附近的主机上,这取决于需求。
\end{itemize}

\item 
当整个数据中心宕机时,如何防止应用程序宕机?
\begin{itemize}
\item 
使用geodes。
\end{itemize}

\item 
为什么使用API网关?
\begin{itemize}
\item 
为了简化客户机代码,不需要硬编码服务实例的地址。
\end{itemize}

\item 
Envoy如何实现各种架构目标?
\begin{itemize}
\item 
它通过提供反压、断路、自动重试和异常值来检测系统的容错。

\item 
它通过允许金丝雀版本和蓝绿色部署来帮助可部署性。

\item 
它提供负载平衡、跟踪、监视和指标。
\end{itemize}
\end{enumerate}