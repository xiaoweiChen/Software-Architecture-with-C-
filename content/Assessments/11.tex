\begin{enumerate}
\item
可以从本章的微基准测试的性能结果中学到什么?
\begin{itemize}
\item 
二分搜索比线性搜索要快得多,即使要检查的元素数量没有那么多。这意味着计算复杂度(又名大O)很重要。可能在你的机器上,即使在最大的数据集上进行最长的二分搜索,仍然比线性搜索的最短搜索要快!

\item 
根据缓存大小的不同,可能还会注意到,当数据不再适合特定的缓存级别时,增加所需的内存会导致速度变慢。
\end{itemize}

\item
如何遍历多维数组对性能重要吗?为什么重要,或为什么不重要?
\begin{itemize}
\item 
这是至关重要的,因为可能会在内存中线性访问数据,这是CPU预取器希望并奖励我们更好的性能,或跳过内存,从而阻碍性能。
\end{itemize}

\item
在协程的例子中,为什么不能在\texttt{do\_routine\_work}函数中创建自己的线程池?
\begin{itemize}
\item 
因为生命周期的问题。
\end{itemize}

\item
如何重新编写协程示例,使其使用生成器,而不仅仅使用任务?
\begin{itemize}
\item 
生成器的主体将需要\texttt{co\_yield}。此外,池中的线程也需要同步,可能需要使用原子线程。
\end{itemize}
\end{enumerate}