\begin{enumerate}
\item
REST式服务的特点是什么?

\begin{itemize}
\item 
显然,有REST API的使用。

\item 
无状态——每个请求包含其处理所需的所有数据。请记住,这并不意味着REST式服务不能使用数据库。

\item 
使用cookie,而不是保持会话
\end{itemize}

\item
可以使用什么工具包来创建自愈分布式体系结构?

\begin{itemize}
\item 
Netflix的猿猴军团
\end{itemize}

\item
微服务应该使用集中存储吗?原因?

\begin{itemize}
\item 
微服务应该使用去中心化存储。每个微服务都应该选择最适合自己的存储类型,因为这样可以提高效率和可扩展性。
\end{itemize}

\item
什么时候应该编写有状态服务,而不是无状态服务?

\begin{itemize}
\item 
无状态不合理,且不需要扩展的时候。例如,当客户端和服务必须保持它们的状态同步时,或者当要发送的状态非常大时。
\end{itemize}

\item
代理与中介拓扑有什么不同?

\begin{itemize}
\item 
中介者在服务之间进行“调解”,因此需要知道如何处理每个请求。代理只知道将每个请求发送到哪里,因此它是一个轻量级组件,可用于创建发布-订阅(发布-订阅)架构。
\end{itemize}

\item
N-tier架构和N-layer架构有什么区别?

\begin{itemize}
\item 
英文里面有两个不同的概念N-Tier和N-Layer,N-Tier指不同系统(一般为不同物理系统)互相协作的架构。而N-Layer指一个系统内部不同模块的结构。N-Tier为物理分层概念,而N-Layer为逻辑分层概念。

\item 
层(Layer)是逻辑的,并指定如何组织代码。层(Tier)是物理的,它指定如何运行代码。每一层都必须由其他层隔开,要么在不同的进程中运行,要么甚至在不同的机器上运行。
\end{itemize}

\item
如何使用基于微服务的架构来取代单体架构?

\begin{itemize}
\item 
增量,在大公司中开辟小型微服务。可以使用在第4章中描述的扼杀器模式。
\end{itemize}
\end{enumerate}