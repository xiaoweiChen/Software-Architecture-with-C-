

\begin{enumerate}
\item 
为什么要关心软件架构?

\begin{itemize}
\item 
架构可以保证实现和维护软件的必要质量。注意和关心它可以防止项目拥有随意架构,从而失去质量,还可以防止软件衰退。
\end{itemize}

\item 
架构师应该成为敏捷团队的最终决策者吗?
\begin{itemize}
\item 
不。敏捷就是授权给整个团队。架构师将他们的经验和知识带到桌面上,但如果决定必须被整个团队接受,那么团队应该对其进行决策,而不仅仅是架构师。利益相关者的需求,在这里也是非常重要的。
\end{itemize}

\item 
SRP与内聚有什么关系?

\begin{itemize}
\item 
遵循SRP会带来更好的凝聚力。如果一个组件开始具有多个职责,通常会变得不那么具有内聚性。这种情况下,最好将其重构为多个组件,每个组件具有单一的职责。通过这种方式,增加了内聚性,因此代码变得更容易理解、开发和维护。
\end{itemize}

\item
项目生命周期的哪些阶段?架构师可以使项目更好?

\begin{itemize}
\item 
从项目开始到项目维护,架构师可以为项目带来价值。在项目开发的早期阶段可以实现最大的价值,因为这是决定项目外观的关键。然而,这并不意味着架构师在开发过程中没有价值,他们可以使项目保持在正确的轨道上。通过帮助决策和监督项目,确保代码不会以随意架构结束,也不会受到软件退化的影响。
\end{itemize}

\item
遵循SRP有什么好处?

\begin{itemize}
\item 
遵循SRP的代码更容易理解和维护,这也意味着它的bug更少。
\end{itemize}

\end{enumerate}