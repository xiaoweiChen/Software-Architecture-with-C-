
至于框架,目前的标准是Google的GTest,与对应的GMock组成了一个小型工具套件,可以遵循C++测试的最佳实践。

GTest/GMock组合的其他替代方案是Catch2、CppUnit和Doctest。CppUnit已经存在很长一段时间了,但缺少更新,所以不推荐使用。Catch2和Doctest都支持现代C++标准——特别是C++14、C++17和C++20。

为了比较这些测试框架,将使用想要测试的代码库。以它为基础,在每个框架中实现测试。

\subsubsubsection{8.3.1\hspace{0.2cm}GTest}

下面是用GTest编写的一个测试示例:

\begin{lstlisting}[style=styleCXX]
#include "customer/customer.h"

#include <gtest/gtest.h>

TEST(basic_responses,
given_name_when_prepare_responses_then_greets_friendly) {
	auto name = "Bob";
	auto code_and_string = responder{}.prepare_response(name);
	ASSERT_EQ(code_and_string.first, web::http::status_codes::OK);
	ASSERT_EQ(code_and_string.second, web::json::value("Hello, Bob!"));
}
\end{lstlisting}

测试期间大部分任务已经抽象,这里主要关注于提供测试的动作(\texttt{prepare\_response})和所需的行为(有\texttt{ASSERT\_EQ}的行)。

\subsubsubsection{8.3.2\hspace{0.2cm}Catch2}

下面是用Catch2编写的一个测试示例:

\begin{lstlisting}[style=styleCXX]
#include "customer/customer.h"

#define CATCH_CONFIG_MAIN // This tells Catch to provide a main() - only do
						  // this in one cpp file
#include "catch2/catch.hpp"

TEST_CASE("Basic responses",
	"Given Name When Prepare Responses Then Greets Friendly") {
	auto name = "Bob";
	auto code_and_string = responder{}.prepare_response(name);
	REQUIRE(code_and_string.first == web::http::status_codes::OK);
	REQUIRE(code_and_string.second == web::json::value("Hello, Bob!"));
}
\end{lstlisting}

看起来和前一个例子很像。有些关键字不同(\texttt{TEST}和\texttt{TEST\_CASE}),检查结果的方式也不同(使用\texttt{REQUIRE(a == b)}而不是\texttt{ASSERT\_EQ(a, b)})。

\subsubsubsection{8.3.3\hspace{0.2cm}CppUnit}

下面是用CppUnit编写的测试示例,将把它分成几个片段。

代码为使用CppUnit库中的构造做了准备工作:

\begin{lstlisting}[style=styleCXX]
#include <cppunit/BriefTestProgressListener.h>
#include <cppunit/CompilerOutputter.h>
#include <cppunit/TestCase.h>
#include <cppunit/TestFixture.h>
#include <cppunit/TestResult.h>
#include <cppunit/TestResultCollector.h>
#include <cppunit/TestRunner.h>
#include <cppunit/XmlOutputter.h>
#include <cppunit/extensions/HelperMacros.h>
#include <cppunit/extensions/TestFactoryRegistry.h>
#include <cppunit/ui/text/TextTestRunner.h>

#include "customer/customer.h"

using namespace CppUnit;
using namespace std;
\end{lstlisting}

接下来,必须定义测试类并实现执行测试用例的方法。之后,注册这个类,以便在测试运行器中使用:

\begin{lstlisting}[style=styleCXX]
class TestBasicResponses : public CppUnit::TestFixture {
	CPPUNIT_TEST_SUITE(TestBasicResponses);
	CPPUNIT_TEST(testBob);
	CPPUNIT_TEST_SUITE_END();
	
protected:
	void testBob();
};

void TestBasicResponses::testBob() {
	auto name = "Bob";
	auto code_and_string = responder{}.prepare_response(name);
	CPPUNIT_ASSERT(code_and_string.first == web::http::status_codes::OK);
	CPPUNIT_ASSERT(code_and_string.second == web::json::value("Hello,
		Bob!"));
}

CPPUNIT_TEST_SUITE_REGISTRATION(TestBasicResponses);
\end{lstlisting}

最后,必须提供测试运行器的行为:

\begin{lstlisting}[style=styleCXX]
int main() {
	CPPUNIT_NS::TestResult testresult;
	
	CPPUNIT_NS::TestResultCollector collectedresults;
	testresult.addListener(&collectedresults);
	CPPUNIT_NS::BriefTestProgressListener progress;
	testresult.addListener(&progress);
	CPPUNIT_NS::TestRunner testrunner;
	testrunner.addTest(CPPUNIT_NS::TestFactoryRegistry::getRegistry().makeTest(
	));
	testrunner.run(testresult);
	
	CPPUNIT_NS::CompilerOutputter compileroutputter(&collectedresults,
	std::cerr);
	compileroutputter.write();
	
	ofstream xmlFileOut("cppTestBasicResponsesResults.xml");
	XmlOutputter xmlOut(&collectedresults, xmlFileOut);
	xmlOut.write();
	
	return collectedresults.wasSuccessful() ? 0 : 1;
}
\end{lstlisting}

与前面的两个示例相比,这里有很多样板文件,而测试看起来与前面的示例非常相似。

\subsubsubsection{8.3.4\hspace{0.2cm}Doctest}

下面是用Doctest编写的一个测试示例:

\begin{lstlisting}[style=styleCXX]
#include "customer/customer.h"

#define DOCTEST_CONFIG_IMPLEMENT_WITH_MAIN
#include <doctest/doctest.h>

TEST_CASE("Basic responses") {
	auto name = "Bob";
	auto code_and_string = responder{}.prepare_response(name);
	REQUIRE(code_and_string.first == web::http::status_codes::OK);
	REQUIRE(code_and_string.second == web::json::value("Hello, Bob!"));
}
\end{lstlisting}

代码很清晰,也很容易理解。Doctest的主要卖点是,与其他具有类似特性的替代产品相比,在编译时和运行时都最快。

\subsubsubsection{8.3.5\hspace{0.2cm}编译时测试}

模板元编程允许编写在编译时执行的C++代码。C++11中添加的\texttt{constexpr}关键字允许开发者使用更多的编译时代码,而C++20中的\texttt{consteval}关键字旨在对代码的计算方式有更大的控制权。

编译时编程的问题是没有简单的方法来进行测试。虽然用于执行时代码的单元测试框架非常丰富(正如刚才看到的),但关于编译时编程的资源并不多。部分原因可能是编译时编程仍然很复杂,而且只针对专家。

尽管如此,不容易并不意味着不可能。就像执行时间测试依赖于在运行时检查断言一样,可以使用\texttt{static\_assert}检查编译时代码的行为,\texttt{static\_assert}在C++11中与\texttt{constexpr}一起引入。

下面是使用\texttt{static\_assert}的简单示例:

\begin{lstlisting}[style=styleCXX]
#include <string_view>
constexpr int generate_lucky_number(std::string_view name) {
	if (name == "Bob") {
		number = number * 7 + static_cast<int>(letter);
	}
	return number;
}

static_assert(generate_lucky_number("Bob") == 808);
\end{lstlisting}

因为可以在编译时计算这里测试的每个值,所以可以使用编译器作为测试框架。











