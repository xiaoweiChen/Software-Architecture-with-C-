
只要测试函数不与外部有太多交互,事情就会很简单很多。当正在测试的单元与第三方组件(如数据库、HTTP连接和特定文件)连接时,问题就出现了。

一方面,希望查看代码在各种环境下的行为。另一方面,不希望等待数据库启动,而且肯定不希望多个数据库包含不同版本的数据,以便检查所有必要条件。

要如何处理这种情况呢?我们不想执行触发所有这些副作用的实际代码,而是使用测试替身。测试替身是代码中模拟实际API的结构,只是它们不执行模拟函数或对象的操作。

最常见的测试替身是mock、fake和stub。虽然它们不相同,但很相似,许多人会认错。

\subsubsubsection{8.4.1\hspace{0.2cm}不同的测试替身}

mock是一种测试替身,注册所有接收到的调用,但仅此而已。不返回任何值,也不会以任何方式改变状态。当有第三方框架来调用代码时,这就很有用。通过使用模拟,可以观察所有调用,从而能够验证框架的行为是否符合预期。

stub的实现稍微复杂一些。具有返回值,这些值已经定义。\texttt{StubRandom.randomInteger()}方法总是返回相同的值(例如,3),这似乎令人惊讶,但当测试返回值的类型或返回值时,可能是正确的stub实现。具体值是什么可能就不那么重要了。

最后,fake是具有工作实现的对象,其行为与实际的生产实现基本相同。区别是,fake数据库可能采取各种快捷方式,避免调用生产数据库或文件系统。

实现命令查询分离(Command Query Separation, CQS)设计模式时,可以使用stub进行双重查询,使用mock进行命令查询。

\subsubsubsection{8.4.3\hspace{0.2cm}测试替身的其他用法}

在检测之外,fake也可以在有限的范围内使用。在内存中处理数据而不需要数据库访问,这对于创建原型或遇到性能瓶颈时也很有用。

\subsubsubsection{8.4.3\hspace{0.2cm}制作测试替身}

为了制作测试替身,通常使用外部库,就像使用单元测试一样。以下是一些主流的解决方案:

\begin{itemize}
\item 
GoogleMock(也被称为gMock),现在是GoogleTest库的一部分: \url{https://github.com/google/googletest}。

\item 
Trompeloeil专注于C++14,很好地集成了许多测试库,如Catch2、doctest和GTest: \url{https://github.com/rollbear/trompeloeil}。
\end{itemize}

下面的代码将向展示如何使用GoogleMock和Trompeloeil。

\hspace*{\fill} \\ %插入空行
\noindent
\textbf{GoogleMock}

由于GoogleMock是GoogleTest的一部分,这里将一起使用:

\begin{lstlisting}[style=styleCXX]
#include "merchants/reviews.h"

#include <gmock/gmock.h>

#include <merchants/visited_merchant_history.h>

#include "fake_customer_review_store.h"

namespace {
	
class mock_visited_merchant : public i_visited_merchant {
public:
	explicit mock_visited_merchant(fake_customer_review_store &store,
									merchant_id_t id)
		: review_store_{store},
		  review_{store.get_review_for_merchant(id).value()} {
		ON_CALL(*this, post_rating).WillByDefault([this](stars s) {
			review_.rating = s;
			review_store_.post_review(review_);
		});
		ON_CALL(*this, get_rating).WillByDefault([this] { return
			review_.rating; });
	}

	MOCK_METHOD(stars, get_rating, (), (override));
	MOCK_METHOD(void, post_rating, (stars s), (override));

private:
	fake_customer_review_store &review_store_;
	review review_;
};

} // namespace

class history_with_one_rated_merchant : public ::testing::Test {
public:
	static constexpr std::size_t CUSTOMER_ID = 7777;
	static constexpr std::size_t MERCHANT_ID = 1234;
	static constexpr const char *REVIEW_TEXT = "Very nice!";
	static constexpr stars RATING = stars{5.f};
	
protected:
	void SetUp() final {
		fake_review_store_.post_review(
		{CUSTOMER_ID, MERCHANT_ID, REVIEW_TEXT, RATING});
		
		// nice mock will not warn on "uninteresting" call to get_rating
		auto mocked_merchant =
			std::make_unique<::testing::NiceMock<mock_visited_merchant>>(
				fake_review_store_, MERCHANT_ID);
				
		merchant_index_ = history_.add(std::move(mocked_merchant));
	}

	fake_customer_review_store fake_review_store_{CUSTOMER_ID};
		history_of_visited_merchants history_{};
		std::size_t merchant_index_{};
	};

TEST_F(history_with_one_rated_merchant,
	   when_user_changes_rating_then_the_review_is_updated_in_store) {
	const auto &mocked_merchant = dynamic_cast<const mock_visited_merchant
&>(
	history_.get_merchant(merchant_index_));
	EXPECT_CALL(mocked_merchant, post_rating);
	
	constexpr auto new_rating = stars{4};
	static_assert(RATING != new_rating);
	history_.rate(merchant_index_, stars{new_rating});
}

TEST_F(history_with_one_rated_merchant,
	   when_user_selects_same_rating_then_the_review_is_not_updated_in_store) {
	const auto &mocked_merchant = dynamic_cast<const mock_visited_merchant
&>(
	history_.get_merchant(merchant_index_));
	EXPECT_CALL(mocked_merchant, post_rating).Times(0);
	
	history_.rate(merchant_index_, stars{RATING});
}
\end{lstlisting}

GTest是本书编写时最流行的C++测试框架,与GMock的集成,意味着GMock在项目中可能已经可用。这种组合使用起来很直观,而且功能齐全,所以如果已经使用了GTest,就没有理由选择其他框架。

\hspace*{\fill} \\ %插入空行
\noindent
\textbf{Trompeloeil}

为了将这个示例与前一个示例进行对比,这次使用Trompeloeil进行测试,并使用Catch2作为测试框架:

\begin{lstlisting}[style=styleCXX]
#include "merchants/reviews.h"

#include "fake_customer_review_store.h"

// order is important
#define CATCH_CONFIG_MAIN
#include <catch2/catch.hpp>
#include <catch2/trompeloeil.hpp>

#include <memory>

#include <merchants/visited_merchant_history.h>

using trompeloeil::_;

class mock_visited_merchant : public i_visited_merchant {
public:
	MAKE_MOCK0(get_rating, stars(), override);
	MAKE_MOCK1(post_rating, void(stars s), override);
};

SCENARIO("merchant history keeps store up to date", "[mobile app]") {
	GIVEN("a history with one rated merchant") {
		static constexpr std::size_t CUSTOMER_ID = 7777;
		static constexpr std::size_t MERCHANT_ID = 1234;
		static constexpr const char *REVIEW_TEXT = "Very nice!";
		static constexpr stars RATING = stars{5.f};
		
		auto fake_review_store_ = fake_customer_review_store{CUSTOMER_ID};
		fake_review_store_.post_review(
			{CUSTOMER_ID, MERCHANT_ID, REVIEW_TEXT, RATING});
		
		auto history_ = history_of_visited_merchants{};
		const auto merchant_index_ =
			history_.add(std::make_unique<mock_visited_merchant>());
			
		auto &mocked_merchant = const_cast<mock_visited_merchant &>(
			dynamic_cast<const mock_visited_merchant &>(
				history_.get_merchant(merchant_index_)));
				
		auto review_ = review{CUSTOMER_ID, MERCHANT_ID, REVIEW_TEXT, RATING};
		ALLOW_CALL(mocked_merchant, post_rating(_))
			.LR_SIDE_EFFECT(review_.rating = _1;
				fake_review_store_.post_review(review_););
	
		ALLOW_CALL(mocked_merchant, get_rating()).LR_RETURN(review_.rating);
		
		WHEN("a user changes rating") {
			constexpr auto new_rating = stars{4};
			static_assert(RATING != new_rating);
			
			THEN("the review is updated in store") {
				REQUIRE_CALL(mocked_merchant, post_rating(_));
				history_.rate(merchant_index_, stars{new_rating});
			}
		}
	
		WHEN("a user selects same rating") {
			THEN("the review is not updated in store") {
				FORBID_CALL(mocked_merchant, post_rating(_));
				history_.rate(merchant_index_, stars{RATING});
			}
		}
	}
}
\end{lstlisting}

Catch2的一个重要特性是,它可以很容易地编写行为驱动开发风格的测试,比如上面这个测试。如果比较喜欢这种风格,那么带有Trompeloeil的Catch2将是一个很好的选择。























