本章集中在测试软件不同部分的体系结构和技术方面。我们了解了测试金字塔,从而了解不同类型的测试,如何对软件项目的整体健康和稳定做出贡献。由于测试可以同时是功能性和非功能性的,我们看到了这两种类型的一些示例。

从本章中最重要的是,测试不是结束。我们需要它们,不是因为它们能带来即时的价值,而是当重构时,或者当我们改变系统现有部分的行为时,可以使用它们来检查已知的回归。当想要进行原因分析时,测试也可以快速验证不同的假设。

建立了理论需求之后,展示了可以用来编写测试替身效果的不同测试框架和库的示例。先编写测试,然后再实现它们需要一些实践,但这种方式是有好处的。这个好处是可以更好对类进行设计。

最后,为了强调现代架构不仅仅是软件代码,还研究了一些用于测试基础设施和部署的工具。下一章中,将看到持续集成和持续部署如何为要架构的应用程序,带来更好的服务质量和健壮性。

