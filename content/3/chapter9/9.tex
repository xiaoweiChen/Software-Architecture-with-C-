目前为止,应该已经了解了在项目开始时实现CI如何从长远来看节省时间。特别是当与CD搭配使用时,还可以减少正在进行的工作。本章中,已经介绍了一些有用的工具,可以实现这两个过程。

上面已经展示了GitLab CI如何在YAML文件中编写流水。已经讨论了代码检查的重要性,并解释了不同形式的代码检查之间的区别。也介绍了Ansible,可以帮助配置管理和创建部署代码。最后,尝试了Packer和Terraform将注意力从创建应用转移到创建系统上。

本章中的知识并不是C++独有的,可以在使用任何语言和任何技术编写的项目中使用。应该记住的是:所有应用程序都需要测试,编译器或静态分析器不足以验证软件。作为架构师,不仅需要考虑当前项目(应用程序本身),还需要考虑产品(应用程序将在其中工作的系统),只交付工作代码是不够的。理解基础设施和部署过程是至关重要的,因为它们是现代系统的新构建块。

下一章将重点介绍软件的安全性。将讨论源代码本身、操作系统级别以及与外部服务和最终用户可能的交互。