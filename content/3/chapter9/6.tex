

In its simplest form, deployment with Ansible may consist of copying a single binary to the target machine and then running that binary. We can achieve this with the following Ansible code:

\begin{tcblisting}{commandshell={}}
tasks:
  # Each Ansible task is written as a YAML object
  # This uses a copy module
  - name: Copy the binaries to the target machine
    copy:
      src: our_application
      dest: /opt/app/bin/our_application
  # This tasks invokes the shell module. The text after the `shell:` key
  # will run in a shell on target machine
  - name: start our application in detached mode
    shell: cd /opt/app/bin; nohup ./our_application </dev/null >/dev/null
2>&1 &
\end{tcblisting}

Every single task starts with a hyphen. For each of the tasks, you need to specify the module it uses (such as the copy module or the shell module), along with its parameters (if applicable). A task may also have a name parameter, which makes it easier to reference the task individually.

