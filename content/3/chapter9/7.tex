
已经到了可以使用本章所学的工具安全地构建CD流水的时候了,已经知道CI如何运作,以及它如何帮助拒绝不适合发布的修改。关于测试自动化的部分展示了拒绝过程中更加健壮的不同方法。通过冒烟测试或端到端测试,可以超越CI,检查整个部署的服务是否满足需求。使用部署代码,不仅可以自动化部署过程,还可以在测试开始失败时准备回滚。

\subsubsubsection{9.7.1\hspace{0.2cm}持续部署和持续交付}

一个巧合是,缩写CD可以表示两种不同的意思。持续交付和持续部署的概念非常相似,但它们有一些细微的区别。本书中,我们关注的是持续部署的概念。这是一种自动化流程,产生于开发者将修改推入中央存储库,并在所有测试都通过的情况下成功地将更改部署到生产环境。因此,可以说这是一个端到端的过程,因为开发人员的工作在没有人工干预的情况下一路传递给客户(当然,要遵循代码审查)。可能听说过与这种方法相关的术语GitOps。由于所有操作都是自动化的,因此推送到Git中指定的分支会触发部署脚本。

持续交付不会走那么远。与CD一样,特点是能够发布最终产品并对其进行测试,但最终产品不会自动交付给客户。可以首先交付给QA,也可以交付给企业内部使用。理想情况下,只要内部客户端接受,交付的工件就可以部署到生产环境中。

\subsubsubsection{9.7.2\hspace{0.2cm}构建一个CD流水示例}

再次以GitLab CI为例,将所有这些技能放在一起来构建流水。在测试步骤之后,将增加两个步骤,一个创建包,另一个使用Ansible部署这个包。

所需要的包装步骤如下:

\begin{tcblisting}{commandshell={}}
# Package the application and publish the artifact
package:
  stage: package
  # Use cpack for packaging
  script:
    - cd build
    - cpack .
  # Save the deb package artifact
  artifacts:
    paths:
      - build/Customer*.deb
\end{tcblisting}

当添加包含工件定义的包步骤时,能够从仪表板下载它们。

这样,就可以在部署步骤中调用Ansible了:

\begin{tcblisting}{commandshell={}}
# Deploy using Ansible
deploy:
  stage: deploy
  script:
    - cd build
    - ansible-playbook -i localhost, ansible.yml
\end{tcblisting}

最后的流水看起来如下所示:

\begin{tcblisting}{commandshell={}}
cache:
  key: all
  paths:
    - .conan
    - build
    
default:
  image: conanio/gcc9

stages:
  - prerequisites
  - build
  - test
  - package
  - deploy
  
before_script:
  - export CONAN_USER_HOME="$CI_PROJECT_DIR"

prerequisites:
  stage: prerequisites
  script:
    - pip install conan==1.34.1
    - conan profile new default || true
    - conan profile update settings.compiler=gcc default
    - conan profile update settings.compiler.libcxx=libstdc++11 default
\end{tcblisting}
\begin{tcblisting}{commandshell={}}
    - conan profile update settings.compiler.version=10 default
    - conan profile update settings.arch=x86_64 default
    - conan profile update settings.build_type=Release default
    - conan profile update settings.os=Linux default
    - conan remote add trompeloeil
https://api.bintray.com/conan/trompeloeil/trompeloeil || true

build:
  stage: build
  script:
    - sudo apt-get update && sudo apt-get install -y docker.io
    - mkdir -p build
    - cd build
    - conan install ../ch08 --build=missing
    - cmake -DBUILD_TESTING=1 -DCMAKE_BUILD_TYPE=Release ../ch08/customer
    - cmake --build .

test:
  stage: test
  script:
    - cd build
    - ctest .
    
# Package the application and publish the artifact
package:
  stage: package
  # Use cpack for packaging
  script:
    - cd build
    - cpack .
# Save the deb package artifact
artifacts:
  paths:
    - build/Customer*.deb

# Deploy using Ansible
deploy:
  stage: deploy
  script:
    - cd build
    - ansible-playbook -i localhost, ansible.yml
\end{tcblisting}

要查看整个示例,请从技术需求部分转到源码的存储库。
