
代码审查既可以用于CI系统,也可以不用于CI系统。他们的主要目的是再次检查代码中引入的每一个更改,以确保它是正确的,符合应用程序的体系结构,并遵循项目的指导方针和最佳实践。

在没有CI系统的情况下使用时,通常是评审人员的任务来手动测试更改,并验证是否按预期工作。CI减少了这种负担,使软件开发人员可以专注于代码的逻辑结构。

\subsubsubsection{9.3.1\hspace{0.2cm}自动控制机制}

自动化测试只是门控机制的一个例子。当质量足够高时,可以保证代码按照设计工作。但是,正确运行的代码和良好的代码之间仍然存在差异。目前为止,已经从本书中了解到,如果代码满足几个条件,就可以认为是好的,功能正确只是其中之一。

其他工具可以实现所需的代码库标准。其中一些已经在前面的章节中介绍过,所以这里就不详细讨论了。记住,在CI/CD流水中使用检查器、代码格式化器和静态分析是一个很好的实践。虽然静态分析可以作为一种门控机制,但可以对进入中央存储库的每个提交应用检测和格式化,使其与其他代码库保持一致。可以在附录中找到更多关于静态代码分析器和格式化工具的信息。

理想情况下,这种机制只需要检查代码是否已经格式化,因为格式化步骤应该由开发人员在将代码放入存储库之前完成。当使用Git作为版本控制系统时,Git hook的机制可以防止在没有运行必要工具的情况下提交代码。

但自动化分析只能帮到这里。接下来,需要检查代码在功能上是否完整,是否没有已知的错误和漏洞,是否符合编码标准,这就需要人工检查了。

\subsubsubsection{9.3.2\hspace{0.2cm}代码审查——手动控制机制}

代码修改的手动检查通常称为代码审查。代码检查的目的是识别问题,包括特定子系统的实现和应用程序的整体架构。自动化性能测试可能会发现,也可能不会发现给定功能的潜在问题。另一方面,人眼通常可以发现问题的次优解决方案。无论是错误的数据结构,还是计算复杂度过高的算法,优秀的架构师都应该能够查明问题所在。

但是,执行代码检查的不仅仅是架构师的角色。同行评审,即由作者的同行执行的代码评审,在开发过程中也有自己的位置。这样的审查之所以有价值,不仅仅是因为允许同事们在彼此的代码中发现错误。更重要的方面是,许多队友可以了解到其他人在做什么。这样,当团队中出现缺勤时(无论是因为长时间的会议、假期,还是工作轮换),另一个团队成员可以代替缺勤的那个人。即使他们不是这个方向的专家,每个其他成员至少知道相关的代码位于哪里,每个人都应该能够记住对代码的最后更改。这既指它们发生的时间,也指这些变化的范围和内容。

随着越来越多的人意识到应用程序内部是如何显示的,他们也更有可能找出一个组件中最近的更改与新发现的bug之间的关联。尽管团队中的每个人可能都有不同的经验,但当每个人都非常了解代码时,就可以共享资源。

因此,代码审查可以检查更改是否符合所需的体系结构,以及实现是否正确。可以将这样的代码评审称为架构评审,或者专家评审。

另一种类型的代码审查,即同行审查,不仅有助于发现错误,而且还提高了团队中其他成员正在做什么的意识。如果有必要,在处理与外部服务集成的更改时,还可以执行不同类型的专家审查。

由于每个接口都可能成为问题来源,接近接口级别的更改应该视为特别危险。这里建议在通常的同行评议中加入来自另一方的专家。例如,如果正在编写生产者的代码,请使用者进行审查。通过这种方式,可以确保不会错过一些重要的用例,这些用例在某个方面可能认为不太可能出现,但另一方面可能经常使用。

\subsubsubsection{9.3.3\hspace{0.2cm}代码评审的不同方法}

代码审查通常会异步进行,这意味着审查变更的作者和审查者之间的交流不是实时发生的。相反,每个参与者在任何时间可以发布他们的评论和建议。当没有更多的评论,作者需要重新修改,并再次将修改置于审查中。这个过程可能需要很多轮,直到每个人对这次修正没有任何意见。

当一个修改特别有争议,并且异步代码审查花费了太多时间时,同步地进行代码审查是有益的。这意味着一场会议(面对面或远程),以解决前进道路上的各种意见。当由于在实施变更时获得的新知识而使变更与最初的决定相矛盾时,这种情况就会发生。

有一些专门的工具只针对代码检查。更常见的是,希望使用内置在存储库服务器中的工具,包括如下服务:

\begin{itemize}
\item 
GitHub

\item 
Bitbucket

\item 
GitLab

\item 
Gerrit
\end{itemize}

上述所有服务都提供Git托管和代码审查。有些甚至更进一步,提供完整的CI/CD管道、问题管理、wiki等等。

当使用代码托管和代码审查的组合包时,默认的工作流程是将更改作为一个单独的分支推送,然后要求项目所有者在一个称为拉请求(或合并请求)的过程中合并更改。尽管名称很花哨,但pull请求或merge请求会告诉项目所有者你有代码想要与主分支合并。这意味着审查人员应该检查更改,以确保一切正常。

\subsubsubsection{9.3.4\hspace{0.2cm}使用拉请求(合并请求)进行代码评审}

像GitLab这样的系统创建拉请求或合并请求非常容易。首先,从命令行将一个新分支推送到中央存储库时,可以观察到以下消息:

\begin{tcblisting}{commandshell={}}
remote:
remote: To create a merge request for fix-ci-cd, visit:
remote:
https://gitlab.com/hosacpp/continuous-integration/merge_requests/new?merge_
request%5Bsource_branch%5D=fix-ci-cd
remote:
\end{tcblisting}

如果以前启用了CI(通过添加.gitlab-ci.yml文件),还会看到新推送的分支已经置于CI进程中。这甚至发生在创建合并请求之前,从而这意味着可以推迟发送给您的同事,直到从CI获得每个自动检查都已通过的信息。

打开合并请求的两种主要方式如下:

\begin{itemize}
\item 
通过链接中提到的推送信息

\item 
通过在GitLab UI中导航到合并请求,并选择创建合并请求按钮或新建合并请求按钮
\end{itemize}

当提交合并请求时,并完成了所有相关字段,将看到CI流水的状态也是可见的。若流水失败,就不可能进行合并。

























