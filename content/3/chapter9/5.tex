
修改经过测试和批准后,现在是时候将其部署到操作环境了。

有许多工具可以帮助进行部署。这里提供Ansible的例子,因为不需要在目标机器上进行任何设置,除了需要安装Python(这是大多数UNIX系统已经拥有的)。为什么是Ansible?它在配置管理领域非常受欢迎,并且得到了值得信赖的开源公司(Red Hat)的支持。

\subsubsubsection{9.5.1\hspace{0.2cm}使用Ansible}

为什么不使用一些已经可用的东西,如Bourne shell脚本或PowerShell?对于简单的部署,shell脚本可能是更好的方法。但随着部署过程变得更加复杂,使用shell的条件语句处理每个可能的初始状态变得更加困难。

与传统的shell脚本不同,处理初始状态之间的差异实际上是Ansible擅长的,使用命令式形式(移动这个文件,编辑那个文件,运行一个特定的命令),Ansible playbooks仅使用声明形式(确保文件在这个路径中可用,确保文件包含指定的行,确保程序正在运行,确保程序成功完成)。

这种声明式方法还有助于实现幂等性。幂等性是函数的一个特征,意味着多次应用该函数将得到与单个应用完全相同的结果。如果Ansible playbook的第一次运行对配置进行了一些更改,那么后续的每次运行都将以所需的状态启动。这将阻止Ansible执行任何更改。

换句话说,当调用Ansible时,它会首先评估配置中所有机器的当前状态:

\begin{itemize}
\item 
如果其中任何一个需要更改,Ansible只会运行所需的任务,以达到所需的状态。

\item 
如果没有必要修改某一特定内容,Ansible是不会去碰它的。只有当期望状态和实际状态不同时,才会看到Ansible采取行动,将实际状态收敛到playbook内容所描述的期望状态。
\end{itemize}

\subsubsubsection{9.5.2\hspace{0.2cm}Ansible如何适应CI/CD流水}

Ansible的幂等性使它成为CI/CD流水中使用的一个目标。毕竟,多次使用相同的Ansible策略没有风险,即使两次运行之间没有变化。如果使用Ansible来编写部署代码,那么创建CD就是为了准备合适的验收测试(比如冒烟测试或端到端测试)。

声明式方法可能需要改变对部署的看法,除了运行playbook,还可以使用Ansible在远程机器上执行一次性命令,但这里不会讨论这个用例,因为它对部署没有实质性帮助。

任何能用shell做的事情,都可以用Ansible的shell模块完成。在playbook中,可以编写任务,指定使用哪些模块以及它们各自的参数。前面提到的shell模块就是这样一个模块,它只是在远程机器上的shell中执行提供的参数。但是使Ansible不仅方便而且跨平台(至少在不同的UNIX发行版中是这样)的原因是它提供了一些模块来操作常见的概念,比如用户管理、包管理和类似的实例。

\subsubsubsection{9.5.3\hspace{0.2cm}创建代码部署}

除了标准库中提供的常规模块外,还有第三方组件允许代码重用。可以单独测试这些组件,这也使部署的代码更加健壮。这样的组件称为角色,包含一组任务,使机器适合承担特定的角色,如Web服务器、数据库或Docker。虽然有些角色让机器准备提供特定服务,但其他角色可能更抽象,例如流行的ansible-hardening角色。这是由OpenStack团队创建的,过使用这个角色保护会让入侵机器变得更加困难。

当开始理解Ansible使用的语言时,所有的playbook都不再只是脚本。反过来,它们将成为部署过程的文档。可以通过运行Ansible逐个使用,或者可以在离线的机器上阅读描述的任务并手动执行所有操作。

团队中使用Ansible进行部署有一个风险,开始就必须确保团队中的每个人都能够使用它并修改相关的任务。DevOps是整个团队必须遵循的实践,它不能只是部分地实现。当应用程序的代码发生重大更改时,需要在部署端进行适当的修改,负责应用程序更改的人员还应该在部署代码中提供修改。当然,这是测试可以验证的,因此门控机制可以拒绝不完整的修改。 

Ansible值得注意的方面是,可以同时运行push和pull模型:

\begin{itemize}
\item 
push模型是在自己的机器或CI系统中运行Ansible的时候。Ansible连接到目标计算机(例如,通过SSH连接),并在目标计算机上执行必要的步骤。

\item 
pull模型中,整个过程由目标机发起。Ansible的组件ansible-pull直接运行在目标机器上,检查代码库以确定特定分支是否有任何更新。在刷新本地playbook后,Ansible照常执行所有步骤。这一次,控制组件和实际执行都发生在同一台机器上。大多数时候,可以周期性地运行ansiblepull,例如从一个cron(定时执行)作业中。
\end{itemize}















