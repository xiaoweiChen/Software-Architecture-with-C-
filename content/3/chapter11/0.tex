
One of the most common reasons to choose C++ as a key programming language for a project is due to performance requirements. C++ has a clear edge over the competition when it comes to performance, but achieving the best results requires understanding relevant problems. This chapter focuses on increasing the performance of C++ software. We'll start by showing you tools for measuring performance. We'll show you a few techniques for increasing single-threaded compute speed. Then we'll discuss how to make use of parallel computing. Finally, we'll show how you can use C++20's coroutines for nonpreemptive multitasking.

本章将讨论以下内容:

\begin{itemize}
\item 
Measuring performance

\item 
Helping the compiler generate performant code

\item 
Parallelizing computations

\item 
Using coroutines
\end{itemize}

First, let's specify what you'll need to run the examples in this chapter.


